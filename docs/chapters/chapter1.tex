\chapter{Wprowadzenie}

Współczesna technologia informatyczna umożliwia pasjonatom IT, administratorom systemów oraz programistom budowanie i zarządzanie własnymi środowiskami testowymi oraz produkcyjnymi w domowych warunkach. Koncepcja HomeLab, czyli prywatnego środowiska IT, zyskuje na popularności dzięki coraz szerszemu dostępowi do wydajnego sprzętu, technologii wirtualizacji oraz narzędzi do automatyzacji zarządzania infrastrukturą. W dobie rosnącej cyfryzacji oraz postępującej popularyzacji rozwiązań chmurowych, coraz więcej osób decyduje się na stworzenie własnego, niezależnego środowiska IT, które może służyć zarówno celom edukacyjnym, jak i profesjonalnym.

Dzięki HomeLab użytkownicy mogą zdobywać praktyczne doświadczenie w administrowaniu serwerami, testować nowe technologie, wdrażać rozwiązania chmurowe oraz konfigurować usługi sieciowe. Środowisko to może również pełnić funkcję osobistego centrum danych, umożliwiającego przechowywanie plików, hostowanie aplikacji czy zarządzanie systemami Internetu Rzeczy (IoT). Wiele osób wykorzystuje HomeLab jako platformę do nauki DevOps, testowania nowych systemów operacyjnych oraz rozwijania umiejętności związanych z cyberbezpieczeństwem.

Jednak dla wielu użytkowników proces konfiguracji i utrzymania takiego środowiska może być skomplikowany i czasochłonny. Wymaga on znajomości technologii wirtualizacji, zarządzania siecią, administracji systemami oraz narzędzi do automatyzacji. Dodatkowym wyzwaniem jest zapewnienie bezpieczeństwa systemu oraz zdalnego dostępu do jego zasobów, co wymaga stosowania odpowiednich mechanizmów uwierzytelniania i autoryzacji.

\section{Cel pracy}

Celem niniejszej pracy magisterskiej jest zaprojektowanie i implementacja systemu HomeLab, który uprości proces budowy, konfiguracji oraz zarządzania własną infrastrukturą IT. System ten ma zapewnić użytkownikom intuicyjne narzędzia do zarządzania serwerami, maszynami wirtualnymi, kontenerami oraz siecią, a także umożliwić zdalny, bezpieczny dostęp do zasobów. Kluczowym założeniem projektu jest maksymalna automatyzacja procesów, co pozwoli na minimalizację konieczności manualnej konfiguracji i zwiększy wygodę użytkowania.

Praca ta ma na celu stworzenie kompleksowego rozwiązania, które pozwoli użytkownikom na:
\begin{itemize}
    \item Proste i szybkie wdrażanie usług i aplikacji w środowisku HomeLab,
    \item Centralne zarządzanie infrastrukturą z poziomu intuicyjnego interfejsu użytkownika,
    \item Automatyzację procesów administracyjnych, takich jak aktualizacje systemów, backupy czy monitorowanie wydajności,
    \item Zapewnienie mechanizmów bezpieczeństwa chroniących zasoby przed nieautoryzowanym dostępem,
    \item Integrację z popularnymi technologiami wykorzystywanymi w branży IT, takimi jak Docker, Kubernetes, Proxmox czy Ansible.
\end{itemize}

Opracowany system ma na celu obniżenie progu wejścia dla użytkowników, którzy chcieliby rozpocząć przygodę z HomeLab, ale nie posiadają zaawansowanej wiedzy technicznej. Jednocześnie zapewni on elastyczność i rozszerzalność, dzięki czemu bardziej doświadczeni użytkownicy będą mogli dostosować go do swoich potrzeb.

\section{Zakres pracy}

W ramach niniejszej pracy zostaną omówione istotne aspekty techniczne związane z budową HomeLab, w tym wybór odpowiednich technologii, metod zarządzania infrastrukturą oraz zapewnienia jej bezpieczeństwa. Ponadto przedstawiona zostanie analiza istniejących rozwiązań oraz uzasadnienie wyboru implementowanych funkcjonalności. Efektem końcowym pracy będzie gotowy system, który może zostać wdrożony przez użytkowników chcących stworzyć własne HomeLab w sposób szybki i efektywny.

Zakres pracy obejmuje następujące obszary:
\begin{itemize}
    \item Analizę wymagań i specyfikację systemu,
    \item Projektowanie i implementację interfejsu użytkownika umożliwiającego łatwe zarządzanie infrastrukturą,
    \item Tworzenie i integrację API odpowiedzialnego za automatyzację operacji administracyjnych,
    \item Implementację mechanizmów bezpieczeństwa, takich jak uwierzytelnianie, autoryzacja oraz ochrona danych,
    \item Przeprowadzenie testów wydajnościowych i funkcjonalnych w celu oceny stabilności oraz optymalizacji systemu,
    \item Przedstawienie teoretycznych aspektów związanych z bezpieczeństwem infrastruktury IT i najlepszymi praktykami w tym zakresie.
\end{itemize}

Niniejsza praca stanowi przyczynek do rozwoju narzędzi dedykowanych osobom zainteresowanym budową i zarządzaniem własnym środowiskiem IT, oferując innowacyjne podejście do automatyzacji i ułatwienia dostępu do HomeLab. W kolejnych rozdziałach zostaną szczegółowo omówione wszystkie kluczowe elementy systemu oraz proces jego implementacji. 

Praca ta nie tylko przedstawia praktyczne rozwiązania ułatwiające zarządzanie infrastrukturą IT, ale także porusza zagadnienia związane z wydajnością, bezpieczeństwem oraz skalowalnością systemów HomeLab. W przyszłości opracowane narzędzie może stać się podstawą do dalszej rozbudowy oraz integracji z innymi technologiami wykorzystywanymi w nowoczesnych centrach danych oraz środowiskach chmurowych.
