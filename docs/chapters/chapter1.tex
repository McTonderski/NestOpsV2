\chapter{Wprowadzenie}

Wraz z dynamicznym rozwojem technologii informatycznych, obserwujemy coraz większe zainteresowanie samodzielnym tworzeniem oraz utrzymywaniem środowisk serwerowych poza scentralizowaną infrastrukturą chmurową. Zjawisko to przybiera szczególne znaczenie wśród pasjonatów IT, administratorów systemów, a także inżynierów oprogramowania, którzy decydują się na budowę tzw. \textit{HomeLabów} – domowych laboratoriów IT. 

HomeLab to prywatne środowisko serwerowe, skonfigurowane najczęściej w warunkach domowych, które umożliwia użytkownikom testowanie, uruchamianie i rozwijanie różnorodnych usług oraz technologii. Może ono przybierać formę jednego serwera z kilkoma kontenerami, klastra urządzeń Raspberry Pi, a nawet rozbudowanego racka z profesjonalnymi serwerami. Niezależnie od skali, HomeLab spełnia istotną rolę edukacyjną, testową, a także produkcyjną w kontekście usług dostępnych lokalnie lub zdalnie poprzez sieć prywatną lub publiczną.

Jednym z głównych powodów, dla których użytkownicy decydują się na stworzenie HomeLaba, są potrzeby edukacyjne i chęć zdobycia doświadczenia z technologiami wykorzystywanymi w środowiskach korporacyjnych. HomeLab stanowi bezpieczne środowisko, w którym można bez ryzyka testować nowe narzędzia, technologie oraz scenariusze awaryjne. Dla wielu użytkowników jest to również sposób na centralizację usług domowych – takich jak serwery multimedialne, automatyczne kopie zapasowe, monitoring, systemy automatyki domowej czy własne rozwiązania chmurowe (tzw. \textit{self-hosting}).

Współczesne HomeLaby korzystają z szerokiej gamy technologii – od wirtualizacji (np. Proxmox, VMware, Hyper-V), przez konteneryzację (Docker, Podman), aż po automatyzację z wykorzystaniem narzędzi takich jak Ansible, Terraform czy Packer. Pojawienie się lekkich systemów operacyjnych, niskonapięciowych jednostek obliczeniowych oraz otwartoźródłowych rozwiązań zarządzających umożliwiło rozwój wydajnych i energooszczędnych środowisk domowych.

Mimo wielu zalet, wdrożenie i zarządzanie HomeLabem nie jest zadaniem trywialnym. Konieczność konfiguracji sieci, wirtualnych maszyn, kontenerów, bezpieczeństwa czy uwierzytelnienia użytkowników może być wyzwaniem, szczególnie dla osób stawiających pierwsze kroki w świecie infrastruktury IT. Dodatkowo, manualna administracja systemem bywa czasochłonna i podatna na błędy, co może negatywnie wpłynąć na stabilność działania usług oraz doświadczenie użytkownika.

W tym kontekście kluczową rolę odgrywa automatyzacja. Dzięki niej możliwe jest ograniczenie liczby powtarzalnych czynności administracyjnych, przyspieszenie wdrożeń oraz zminimalizowanie ryzyka błędów konfiguracyjnych. Automatyzacja umożliwia również realizację bardziej zaawansowanych scenariuszy, takich jak:
\begin{itemize}
    \item samonaprawiające się klastry,
    \item dynamiczne skalowanie zasobów,
    \item automatyczne aktualizacje i testy regresji,
    \item ciągła integracja i dostarczanie (CI/CD),
    \item monitorowanie i alertowanie.
\end{itemize}

Dzięki automatyzacji, HomeLab może stać się nie tylko narzędziem edukacyjnym, ale także realnym środowiskiem produkcyjnym obsługującym usługi użytkownika w sposób niezawodny, elastyczny i bezpieczny.

\section{Cel pracy}

Celem niniejszej pracy magisterskiej jest zaprojektowanie i implementacja nowoczesnego systemu zarządzania środowiskiem HomeLab, który będzie wspierał użytkownika w procesie budowy, konfiguracji i obsługi infrastruktury IT w sposób zautomatyzowany, intuicyjny oraz skalowalny. 

Proponowane rozwiązanie ma za zadanie:
\begin{itemize}
    \item Ułatwić wdrażanie i zarządzanie usługami w kontenerach,
    \item Zapewnić dostęp do intuicyjnego interfejsu użytkownika umożliwiającego kontrolę nad całym środowiskiem,
    \item Zminimalizować potrzebę ręcznej ingerencji w konfigurację systemów,
    \item Wspierać integrację z popularnymi rozwiązaniami open-source (np. Docker, SQLite, TailScale),
    \item Zwiększyć bezpieczeństwo i kontrolę nad uruchamianymi usługami.
\end{itemize}

System ma na celu obniżenie progu wejścia dla osób rozpoczynających pracę z HomeLabem oraz dostarczenie bardziej zaawansowanym użytkownikom elastycznej i rozszerzalnej platformy do dalszego rozwoju. Całość zostanie udostępniona jako projekt open-source, co umożliwi społeczności jego dalsze rozwijanie i dostosowywanie.

\section{Zakres pracy}

W ramach niniejszej pracy zostaną omówione następujące zagadnienia:
\begin{itemize}
    \item Analiza dostępnych technologii oraz przegląd istniejących rozwiązań open-source w zakresie zarządzania infrastrukturą domową,
    \item Projektowanie architektury systemu, obejmującej backend, frontend oraz warstwę automatyzującą,
    \item Implementacja interfejsu użytkownika umożliwiającego zdalne zarządzanie HomeLabem,
    \item Budowa API służącego do komunikacji z systemem operacyjnym i usługami backendowymi,
    \item Integracja z kontenerami Docker oraz obsługa uruchamiania, zatrzymywania i monitorowania usług,
    \item Wdrożenie mechanizmów bezpieczeństwa: autoryzacji, uwierzytelnienia i ochrony dostępu,
    \item Przeprowadzenie testów funkcjonalnych oraz wydajnościowych na urządzeniu Raspberry Pi 5,
    \item Udokumentowanie i przygotowanie instalatora umożliwiającego łatwe wdrożenie systemu przez użytkownika końcowego.
\end{itemize}

Opracowane rozwiązanie nie tylko upraszcza proces zarządzania domową infrastrukturą IT, ale stanowi również punkt wyjścia do dalszej rozbudowy. W kolejnych rozdziałach omówione zostaną szczegółowo zarówno decyzje projektowe, jak i konkretne aspekty implementacyjne oraz propozycje dalszego rozwoju systemu. Projekt publikowany jest w repozytorium GitHub, co umożliwia społeczności swobodne korzystanie oraz modyfikowanie aplikacji zgodnie z własnymi potrzebami.
