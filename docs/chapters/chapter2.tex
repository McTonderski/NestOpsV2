\chapter{Czym jest HomeLab oraz analiza istniejących rozwiązań}

\section{Definicja HomeLab oraz znaczenie}
HomeLab jest prywatnym środowiskiem IT, dzięki któremu entuzjaści nowych technologii, administratorzy systemów oraz programiści mogą w lokalnym – domowym środowisku testować, rozwijać oraz zarządzać własną infrastrukturą IT. Jego głównym zamierzeniem jest stworzenie realistycznego środowiska do eksperymentowania z technologiami chmurowymi, wirtualizacją, konteneryzacją oraz narzędziami DevOps. Własny system HomeLab to również metoda na rezygnację z komercyjnych subskrypcji, takich jak Google Drive, Dropbox czy OneDrive, co pozwala na pełną kontrolę nad dostępem do prywatnych danych. Dzięki niemu zwiększa się prywatność poprzez wyeliminowanie potrzeby przechowywania zdjęć w usługach chmurowych, takich jak Google Photos.

HomeLaby znajdują zastosowanie w wielu obszarach, w tym:
\begin{itemize}
    \item nauka administracji serwerami i sieciami,
    \item testowanie nowych technologii przed użyciem ich w środowisku produkcyjnym,
    \item budowanie prywatnej chmury oraz rozwiązań do przechowywania danych,
    \item analiza bezpieczeństwa i przeprowadzanie testów penetracyjnych,
    \item tworzenie automatyzacji dla infrastruktury IT,
    \item uniezależnienie się od komercyjnych dostawców chmury w celu zwiększenia kontroli nad własnymi danymi.
\end{itemize}

\section{Technologie wykorzystywane w HomeLabach}
HomeLab może składać się z różnych komponentów, od dedykowanych serwerów fizycznych po rozwiązania chmurowe i kontenerowe. Kluczowe technologie wykorzystywane w HomeLabach obejmują:

\subsection{Wirtualizacja i konteneryzacja}
\begin{itemize}
    \item Proxmox VE – platforma do zarządzania maszynami wirtualnymi i kontenerami.
    \item VMware ESXi – profesjonalne narzędzie do wirtualizacji serwerów.
    \item Hyper-V – narzędzie do wirtualizacji dostarczane przez Microsoft wraz z systemem Windows.
    \item Docker i Kubernetes – technologie konteneryzacji, pozwalające na elastyczne zarządzanie aplikacjami i zasobami.
\end{itemize}

\subsection{Automatyzacja i zarządzanie konfiguracją}
\begin{itemize}
    \item Ansible, Terraform, Puppet, Chef – narzędzia do automatyzacji wdrażania i zarządzania infrastrukturą.
\end{itemize}

\subsection{Monitoring i analiza}
\begin{itemize}
    \item Prometheus i Grafana – rozwiązania do monitorowania wydajności i wizualizacji danych.
    \item Zabbix – platforma do monitorowania infrastruktury IT.
\end{itemize}

\section{Analiza istniejących systemów do zarządzania homelabem}
\subsection{Przegląd dostępnych rozwiązań}
Na rynku istnieje kilka systemów umozliwiajacych zarzadzanie homelabem. Do najpopularniejszych nalezą:
\begin{itemize}
    \item Proxmox VE \cite{Proxmox} - rozbudowany, open-source rozwiązanie do zarządzania maszynami wirtualnymi i kontenerami, oferujące intrgeację z Ceph i wysoką dostępność.
    \item Unraid \cite{Unraid} - popularne rozwiązanie NAS z obsługą wirtualizacji i kontenerów, cenione za łatwość obsługi ale ograniczone zastosowanie korporacyjne.
    \item OpenStack \cite{OpenStack} - potęzna platforma chmurowa, która moze być uzywana do zarządzania homelabem, ale jej skomplikowana konfiguracja sprawia, ze nie jest przyjazna dla poczatkujacych uzytkowników.
    \item TrueNAS \cite{TrueNAS} - rozbudowane oprogramowanie do zarządzania przestrzenią dyskową, które umozliwa tworzenie prywatnych chmur danych
    \item Docker \cite{Docker} + Kubernetes \cite{Kubernetes} - stosowane w bardziej zaawansowanych wdrozeniach do zarzadzania kontenerami, ale wymagające większej wiedzy technicznej.
\end{itemize}
\subsection{Zalety i ograniczenia konkurencyjnych systemów}

\subsubsection{Proxmox VE \cite{Proxmox}}
\begin{minipage}{0.45\textwidth}
    Zalety
    \begin{itemize}
        \item Darmowa wersja open-source.
        \item Wsparcie dla maszyn wirtualnych (KVM) i kontenerów (LXC).
        \item Mozliwość tworzenia klastrów wysokiej dostępności.
    \end{itemize}
\end{minipage}\hfil
\begin{minipage}{0.45\textwidth}
    Wady
    \begin{itemize}
        \item Brak pełnej automatyzacji wdrozeń.
        \item Stosunkowo wysoki próg wejścia dla początkujących uzytkowników.
    \end{itemize}
\end{minipage}

\subsubsection{Unraid \cite{Unraid}}
\begin{minipage}{0.45\textwidth}
    Zalety
    \begin{itemize}
        \item Intuicyjny interfejs uzytkownika.
        \item Łatwa obsługa pamięci masowej i kontrolerów.
    \end{itemize}
\end{minipage}\hfil
\begin{minipage}{0.45\textwidth}
    Wady
    \begin{itemize}
        \item Model licencyjny oparty na opłacie jednorazowej.
        \item Ograniczona integracja z systemami chmurowymi.
    \end{itemize}
\end{minipage}

\subsubsection{OpenStack \cite{OpenStack}}
\begin{minipage}{0.45\textwidth}
    Zalety
    \begin{itemize}
        \item Zaawansowane funkcje chmurowe.
        \item Skalowalność i modularność.
    \end{itemize}
\end{minipage}\hfil
\begin{minipage}{0.45\textwidth}
    Wady
    \begin{itemize}
        \item Bardzo wysoka trudność wdrozenia.
        \item Wymaga duzej ilości zasobów sprzętowych.
    \end{itemize}
\end{minipage}

\subsubsection{TrueNAS \cite{TrueNAS}}
\begin{minipage}{0.45\textwidth}
    Zalety
    \begin{itemize}
        \item Silne wsparcie dla przechowywania danych.
        \item Wbudowana replikacja i ochrona RAID
    \end{itemize}
\end{minipage}\hfil
\begin{minipage}{0.45\textwidth}
    Wady
    \begin{itemize}
        \item Skupione głównie na funkcjach NAS.
        \item Brak natywnego wsparcia dla maszyn wirtualnych.
    \end{itemize}
\end{minipage}


\subsubsection{Docker \cite{Docker} + Kubernetes \cite{Kubernetes}}
\begin{minipage}{0.45\textwidth}
    Zalety
    \begin{itemize}
        \item Elastyczność w zarządzaniu aplikacjami kontenerowymi.
        \item Łatwe skalowanie infrastruktury.
    \end{itemize}
\end{minipage}\hfil
\begin{minipage}{0.45\textwidth}
    Wady
    \begin{itemize}
        \item Wymaga duzej wiedzy technicznej.
        \item Brak wsparcia dla maszyn wirtualnych.
    \end{itemize}
\end{minipage}


\subsection{Identyfikacja luki technologicznej}
Analiza powyższego porównania dostępnych systemów pokazuje, że żadne z obecnych rozwiązań nie zapewnia jednocześnie:
\begin{itemize}
    \item Pełnej integracji zarządzania maszynami wirtualnymi, kontenerami i przestrzenią dyskową w jednym ekosystemie.
    \item Prostego i intuicyjnego interfejsu dla użytkowników niebędących ekspertami w zarządzaniu infrastrukturą IT.
    \item Natychmiastowej automatyzacji wdrażania, bez konieczności skomplikowanej konfiguracji narzędzi DevOps.
    \item Wbudowanej funkcjonalności związanej z bezpieczeństwem i prywatnością, eliminującej konieczność korzystania z komercyjnych rozwiązań chmurowych.
\end{itemize}

Proponowany system HomeLab ma na celu uzupełnienie tej luki poprzez stworzenie intuicyjnego narzędzia do zarządzania domową infrastrukturą IT, które zapewni łatwość obsługi, pełną automatyzację oraz zwiększoną prywatność użytkowników.
