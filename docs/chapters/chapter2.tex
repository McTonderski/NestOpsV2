\chapter{Czym jest HomeLab oraz analiza istniejących rozwiązań}

\section{Definicja HomeLab oraz znaczenie}
HomeLab jest prywatnym środowiskiem IT, dzięki któremu entuzjaści nowych technologii, administratorzy systemów oraz programiści mogą w lokalnym – domowym środowisku testować, rozwijać oraz zarządzać własną infrastrukturą IT. Jego głównym zamierzeniem jest stworzenie realistycznego środowiska do eksperymentowania z technologiami chmurowymi, wirtualizacją, konteneryzacją oraz narzędziami DevOps. Własny system HomeLab to również metoda na rezygnację z komercyjnych subskrypcji, takich jak Google Drive, Dropbox czy OneDrive, co pozwala na pełną kontrolę nad dostępem do prywatnych danych. Dzięki niemu zwiększa się prywatność poprzez wyeliminowanie potrzeby przechowywania zdjęć w usługach chmurowych, takich jak Google Photos.

HomeLaby znajdują zastosowanie w wielu obszarach, w tym:
\begin{itemize}
    \item nauka administracji serwerami i sieciami,
    \item testowanie nowych technologii przed użyciem ich w środowisku produkcyjnym,
    \item budowanie prywatnej chmury oraz rozwiązań do przechowywania danych,
    \item analiza bezpieczeństwa i przeprowadzanie testów penetracyjnych,
    \item tworzenie automatyzacji dla infrastruktury IT,
    \item uniezależnienie się od komercyjnych dostawców chmury w celu zwiększenia kontroli nad własnymi danymi.
\end{itemize}

\subsection{Ewolucja HomeLabów}
Historia HomeLabów sięga czasów, gdy entuzjaści IT zaczynali od prostych zestawów serwerów fizycznych w domach, często z pojedynczymi maszynami do nauki i testów. Początkowo były to głównie serwery oparte na systemach Linux lub Windows Server, służące do eksperymentów z sieciami, usługami plików czy prostymi aplikacjami. Wraz z rozwojem technologii wirtualizacji, takich jak VMware czy Hyper-V, HomeLaby zaczęły ewoluować, umożliwiając uruchamianie wielu maszyn wirtualnych na jednym fizycznym serwerze, co znacznie zwiększyło możliwości testowe i oszczędność zasobów.

Kolejnym etapem było pojawienie się konteneryzacji, przede wszystkim dzięki Dockerowi, która pozwoliła na lekkie i szybkie uruchamianie aplikacji w izolowanych środowiskach.

Docker został zaprezentowany w 2013 roku jako projekt firmy dotCloud, która później zmieniła nazwę na Docker Inc. Technologia ta wywodzi się z systemu konteneryzacji opierającego się na mechanizmach LXC (Linux Containers), które były dostępne w jądrze Linuxa już wcześniej, ale trudne w użyciu. Docker wprowadził prosty interfejs do tworzenia, uruchamiania i zarządzania kontenerami, co zrewolucjonizowało sposób dystrybucji i uruchamiania aplikacji. W kolejnych latach Docker zyskał ogromną popularność, stając się de facto standardem w konteneryzacji, a jego rozwój obejmował m.in. wprowadzenie Docker Swarm (2015) – własnego systemu orkiestracji, integrację z chmurami publicznymi oraz rozwój narzędzi do zarządzania cyklem życia kontenerów. W 2017 roku Docker przekazał część swojej technologii do fundacji CNCF w postaci silnika kontenerowego containerd, a sam projekt coraz częściej był wykorzystywany w połączeniu z Kubernetesem, który stał się dominującym rozwiązaniem orkiestracyjnym.

Wprowadzenie orkiestratorów kontenerów, takich jak Kubernetes, umożliwiło zarządzanie złożonymi systemami rozproszonymi nawet w domowym środowisku. Obecnie HomeLaby coraz częściej korzystają z podejścia cloud-native, integrując narzędzia infrastruktury jako kod (IaC), takie jak Terraform czy Ansible, oraz praktyki GitOps, gdzie konfiguracje i wdrożenia są zarządzane poprzez repozytoria Git, co zwiększa powtarzalność i automatyzację.

Ta ewolucja spowodowała, że dzisiejsze HomeLaby to nie tylko miejsca do nauki, ale pełnoprawne środowiska produkcyjne, które mogą obsługiwać usługi multimedialne, automatyzację domową, a nawet prywatne chmury danych i aplikacji.

\section{Technologie wykorzystywane w HomeLabach}

W niniejszej sekcji przedstawiono wybrane technologie i rozwiązania wykorzystywane w środowiskach typu HomeLab. Należy jednak zaznaczyć, że poniższy przegląd ma charakter poglądowy i służy jedynie przedstawieniu gotowych rozwiązań dostępnych na rynku. Opisane narzędzia i systemy mają na celu pogłębienie wiedzy czytelnika w zakresie możliwości technologicznych, ale w większości nie zostały wykorzystane bezpośrednio w implementacji systemu zaprojektowanego w ramach niniejszej pracy.

HomeLab może składać się z różnych komponentów, od dedykowanych serwerów fizycznych po rozwiązania chmurowe i kontenerowe. Kluczowe technologie wykorzystywane w HomeLabach obejmują:

\subsection{Wirtualizacja i konteneryzacja}
\begin{itemize}
    \item \textbf{Proxmox VE} - platforma do zarządzania maszynami wirtualnymi i kontenerami.
    
    Przykład tworzenia maszyny wirtualnej za pomocą CLI Proxmox:
    \begin{verbatim}
    qm create 100 --memory 2048 --net0 virtio,bridge=vmbr0 --cdrom /var/lib/vz/template/iso/ubuntu.iso
    qm start 100
    \end{verbatim}

    \item \textbf{VMware ESXi} - profesjonalne narzędzie do wirtualizacji serwerów.
    
    Przykład automatycznego wdrożenia VM z wykorzystaniem PowerCLI:
    \begin{verbatim}
    New-VM -Name "TestVM" -ResourcePool "Resources" -Datastore "Datastore1" -MemoryGB 4 -NumCPU 2
    \end{verbatim}

    \item \textbf{Hyper-V} - narzędzie do wirtualizacji dostarczane przez Microsoft wraz z systemem Windows.
    
    Przykład tworzenia maszyny wirtualnej PowerShell:
    \begin{verbatim}
    New-VM -Name "LabVM" -MemoryStartupBytes 2GB -Generation 2 -NewVHDPath "C:\VMs\LabVM.vhdx" -NewVHDSizeBytes 40GB
    \end{verbatim}

    \item \textbf{Docker i Kubernetes} - technologie konteneryzacji, pozwalające na elastyczne zarządzanie aplikacjami i zasobami.
    
    Przykład prostego pliku \texttt{docker-compose.yaml}:
    \begin{verbatim}
    version: '3'
    services:
      web:
        image: nginx:latest
        ports:
          - "80:80"
    \end{verbatim}
    
    Przykład uruchomienia playbooka Kubernetes:
    \begin{verbatim}
    kubectl apply -f deployment.yaml
    \end{verbatim}
\end{itemize}

\subsection{Automatyzacja i zarządzanie konfiguracją}
\begin{itemize}
    \item \textbf{Ansible, Terraform, Puppet, Chef} – narzędzia do automatyzacji wdrażania i zarządzania infrastrukturą.

    Przykład użycia Ansible do instalacji Nginx:
    \begin{verbatim}
    - hosts: homelab_servers
      become: yes
      tasks:
        - name: Install Nginx
          apt:
            name: nginx
            state: present
    \end{verbatim}

    Przykład Terraform do stworzenia maszyny wirtualnej na Proxmox:
    \begin{verbatim}
    resource "proxmox_vm_qemu" "vm1" {
      name   = "homelab-vm"
      memory = 2048
      cores  = 2
      disk {
        size = "20G"
      }
    }
    \end{verbatim}
\end{itemize}

\subsection{Monitoring i analiza}
\begin{itemize}
    \item \textbf{Prometheus i Grafana} – rozwiązania do monitorowania wydajności i wizualizacji danych.
    
    Przykład konfiguracji Prometheus scrape config:
    \begin{verbatim}
    scrape_configs:
      - job_name: 'node_exporter'
        static_configs:
          - targets: ['localhost:9100']
    \end{verbatim}
    
    \item \textbf{Zabbix} – platforma do monitorowania infrastruktury IT.
    
    Przykład prostego szablonu Zabbix do monitorowania serwera:
    \begin{verbatim}
    Hostname: homelab-server
    Items:
      - CPU load
      - Memory usage
      - Disk space
    \end{verbatim}
\end{itemize}

\subsection{Bezpieczeństwo i ochrona danych}
Bezpieczeństwo jest kluczowym aspektem każdego HomeLaba, zwłaszcza gdy przechowywane są dane prywatne. Wśród najważniejszych technologii i praktyk znajdują się:

\begin{itemize}
    \item \textbf{VPN} – umożliwia bezpieczne połączenie z domową siecią z dowolnego miejsca. Popularne rozwiązania to OpenVPN, WireGuard.
    
    Przykład podstawowej konfiguracji WireGuard:
    \begin{verbatim}
    [Interface]
    PrivateKey = <private_key>
    Address = 10.0.0.1/24

    [Peer]
    PublicKey = <peer_public_key>
    AllowedIPs = 10.0.0.2/32
    \end{verbatim}

    \item \textbf{Zapory sieciowe (firewalle)} – kontrolują ruch sieciowy, chroniąc przed nieautoryzowanym dostępem. Można używać iptables, ufw lub dedykowanych urządzeń.

    Przykład reguły ufw zezwalającej na ruch SSH:
    \begin{verbatim}
    sudo ufw allow ssh
    sudo ufw enable
    \end{verbatim}

    \item \textbf{Strategie backupu} – regularne tworzenie kopii zapasowych danych i konfiguracji, np. za pomocą rsync, BorgBackup, czy dedykowanych narzędzi chmurowych.

    Przykład backupu katalogu za pomocą rsync:
    \begin{verbatim}
    rsync -av --delete /home/user/data /mnt/backup/
    \end{verbatim}

    \item \textbf{Szyfrowanie danych} – zarówno na poziomie dysków (LUKS, BitLocker), jak i podczas transmisji (TLS).

    Przykład szyfrowania dysku z LUKS:
    \begin{verbatim}
    cryptsetup luksFormat /dev/sdX
    cryptsetup open /dev/sdX encrypted_disk
    \end{verbatim}
\end{itemize}

\section{Analiza istniejących systemów do zarządzania homelabem}
\subsection{Przegląd dostępnych rozwiązań}
Na rynku istnieje kilka systemów umożliwiających zarządzanie homelabem. Do najpopularniejszych należą:
\begin{itemize}
    \item Proxmox VE \cite{Proxmox} - rozbudowane, open-source rozwiązanie do zarządzania maszynami wirtualnymi i kontenerami, oferujące integrację z Ceph i wysoką dostępność.
    \item Unraid \cite{Unraid} - popularne rozwiązanie NAS z obsługą wirtualizacji i kontenerów, cenione za łatwość obsługi ale ograniczone zastosowanie korporacyjne.
    \item OpenStack \cite{OpenStack} - potężna platforma chmurowa, która może być używana do zarządzania homelabem, ale jej skomplikowana konfiguracja sprawia, że nie jest przyjazna dla początkujących użytkowników.
    \item TrueNAS \cite{TrueNAS} - rozbudowane oprogramowanie do zarządzania przestrzenią dyskową, które umożliwia tworzenie prywatnych chmur danych.
    \item Docker \cite{Docker} + Kubernetes \cite{Kubernetes} - stosowane w bardziej zaawansowanych wdrożeniach do zarządzania kontenerami, ale wymagające większej wiedzy technicznej.
    \item CasaOS \cite{CasaOS} - nowoczesny, łatwy w obsłudze system do zarządzania usługami domowymi i multimedialnymi, skierowany do użytkowników z mniejszym doświadczeniem technicznym.
\end{itemize}

\subsection{Zalety i ograniczenia konkurencyjnych systemów}

\subsubsection{Proxmox VE \cite{Proxmox}}
\begin{minipage}{0.45\textwidth}
    Zalety
    \begin{itemize}
        \item Darmowa wersja open-source.
        \item Wsparcie dla maszyn wirtualnych (KVM) i kontenerów (LXC).
        \item Możliwość tworzenia klastrów wysokiej dostępności.
        \item Przykład zastosowania: idealny do środowisk testowych oraz małych firm wymagających elastycznej wirtualizacji.
    \end{itemize}
\end{minipage}\hfil
\begin{minipage}{0.45\textwidth}
    Wady
    \begin{itemize}
        \item Brak pełnej automatyzacji wdrożeń.
        \item Stosunkowo wysoki próg wejścia dla początkujących użytkowników.
        \item Przykład ograniczenia: mniej przyjazny dla użytkowników nieznających CLI lub skryptów.
    \end{itemize}
\end{minipage}

\subsubsection{Unraid \cite{Unraid}}
\begin{minipage}{0.45\textwidth}
    Zalety
    \begin{itemize}
        \item Intuicyjny interfejs użytkownika.
        \item Łatwa obsługa pamięci masowej i kontrolerów.
        \item Przykład zastosowania: idealne do mediów domowych i prostych rozwiązań NAS.
    \end{itemize}
\end{minipage}\hfil
\begin{minipage}{0.45\textwidth}
    Wady
    \begin{itemize}
        \item Model licencyjny oparty na opłacie jednorazowej.
        \item Ograniczona integracja z systemami chmurowymi.
        \item Przykład ograniczenia: mniej elastyczny w środowiskach korporacyjnych.
    \end{itemize}
\end{minipage}

\subsubsection{OpenStack \cite{OpenStack}}
\begin{minipage}{0.45\textwidth}
    Zalety
    \begin{itemize}
        \item Zaawansowane funkcje chmurowe.
        \item Skalowalność i modularność.
        \item Przykład zastosowania: odpowiedni dla dużych środowisk testowych i produkcyjnych.
    \end{itemize}
\end{minipage}\hfil
\begin{minipage}{0.45\textwidth}
    Wady
    \begin{itemize}
        \item Bardzo wysoka trudność wdrożenia.
        \item Wymaga dużej ilości zasobów sprzętowych.
        \item Przykład ograniczenia: niepraktyczny dla małych i średnich HomeLabów.
    \end{itemize}
\end{minipage}

\subsubsection{TrueNAS \cite{TrueNAS}}
\begin{minipage}{0.45\textwidth}
    Zalety
    \begin{itemize}
        \item Silne wsparcie dla przechowywania danych.
        \item Wbudowana replikacja i ochrona RAID.
        \item Przykład zastosowania: idealny do przechowywania dużych zbiorów danych i backupów.
    \end{itemize}
\end{minipage}\hfil
\begin{minipage}{0.45\textwidth}
    Wady
    \begin{itemize}
        \item Skupione głównie na funkcjach NAS.
        \item Brak natywnego wsparcia dla maszyn wirtualnych.
        \item Przykład ograniczenia: ograniczona funkcjonalność w zakresie wirtualizacji.
    \end{itemize}
\end{minipage}

\subsubsection{Docker \cite{Docker} + Kubernetes \cite{Kubernetes}}
\begin{minipage}{0.45\textwidth}
    Zalety
    \begin{itemize}
        \item Elastyczność w zarządzaniu aplikacjami kontenerowymi.
        \item Łatwe skalowanie infrastruktury.
        \item Przykład zastosowania: idealne dla deweloperów aplikacji mikroserwisowych.
    \end{itemize}
\end{minipage}\hfil
\begin{minipage}{0.45\textwidth}
    Wady
    \begin{itemize}
        \item Wymaga dużej wiedzy technicznej.
        \item Brak wsparcia dla maszyn wirtualnych.
        \item Przykład ograniczenia: nieodpowiednie dla użytkowników początkujących.
    \end{itemize}
\end{minipage}

\subsubsection{CasaOS \cite{CasaOS}}
\begin{minipage}{0.45\textwidth}
    Zalety
    \begin{itemize}
        \item Bardzo prosty, intuicyjny interfejs użytkownika.
        \item Łatwa instalacja i konfiguracja.
        \item Skupienie na usługach multimedialnych i domowych.
        \item Przykład zastosowania: idealne dla użytkowników domowych chcących zarządzać multimediami i prostymi usługami.
    \end{itemize}
\end{minipage}\hfil
\begin{minipage}{0.45\textwidth}
    Wady
    \begin{itemize}
        \item Ograniczona skalowalność i funkcjonalność w porównaniu do rozwiązań enterprise.
        \item Brak zaawansowanych funkcji automatyzacji i integracji.
        \item Przykład ograniczenia: nie nadaje się do rozbudowanych środowisk IT.
    \end{itemize}
\end{minipage}

\subsection{Identyfikacja luki technologicznej}
Analiza powyższego porównania dostępnych systemów pokazuje, że żadne z obecnych rozwiązań nie zapewnia jednocześnie:
\begin{itemize}
    \item Pełnej integracji zarządzania maszynami wirtualnymi, kontenerami i przestrzenią dyskową w jednym ekosystemie.
    \item Prostego i intuicyjnego interfejsu dla użytkowników niebędących ekspertami w zarządzaniu infrastrukturą IT.
    \item Natychmiastowej automatyzacji wdrażania, bez konieczności skomplikowanej konfiguracji narzędzi DevOps.
    \item Wbudowanej funkcjonalności związanej z bezpieczeństwem i prywatnością, eliminującej konieczność korzystania z komercyjnych rozwiązań chmurowych.
\end{itemize}

Proponowany system HomeLab ma na celu uzupełnienie tej luki poprzez stworzenie intuicyjnego narzędzia do zarządzania domową infrastrukturą IT, które zapewni łatwość obsługi, pełną automatyzację oraz zwiększoną prywatność użytkowników.

\subsection{Podsumowanie rozdziału}
W niniejszym rozdziale przedstawiono definicję HomeLaba, jego znaczenie oraz ewolucję od prostych serwerów fizycznych do nowoczesnych, zautomatyzowanych i cloud-native środowisk. Omówiono kluczowe technologie wykorzystywane w HomeLabach, w tym wirtualizację, konteneryzację, automatyzację, monitoring oraz aspekty bezpieczeństwa. Analiza istniejących systemów do zarządzania HomeLabem wykazała, że chociaż dostępne rozwiązania oferują szeroki zakres funkcji, żadne z nich nie łączy kompleksowo prostoty obsługi, pełnej integracji oraz automatyzacji, jednocześnie dbając o bezpieczeństwo i prywatność użytkowników.

W związku z tym istnieje wyraźna luka technologiczna, którą proponowany system HomeLab ma wypełnić. Nowa platforma ma za zadanie dostarczyć użytkownikom narzędzie łatwe w obsłudze, pozwalające na kompleksowe zarządzanie infrastrukturą – od maszyn wirtualnych, przez kontenery, po przestrzeń dyskową – z automatyzacją wdrożeń i wbudowanymi mechanizmami bezpieczeństwa. Takie podejście umożliwi zarówno początkującym, jak i zaawansowanym użytkownikom efektywne korzystanie z domowej infrastruktury IT, zwiększając jednocześnie kontrolę nad danymi i minimalizując ryzyko związane z ich przechowywaniem w chmurze komercyjnej.
