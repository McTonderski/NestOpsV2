\chapter{Testowanie i analiza systemu}

\section{Testy jednostkowe i integracyjne}
Testowanie oprogramowania jest kluczowym elementem zapewnienia jego jakości, stabilności i niezawodności. W ramach niniejszej pracy zastosowano zarówno testy jednostkowe, jak i testy integracyjne w celu weryfikacji poprawności działania poszczególnych modułów systemu oraz ich wzajemnych interakcji.

\subsection{Testy jednostkowe}

Testy jednostkowe koncentrują się na sprawdzaniu poprawności działania pojedynczych funkcji i metod w izolacji. Ich głównym celem jest szybkie wykrywanie błędów w logice aplikacji oraz zapewnienie, że każdy komponent działa zgodnie z oczekiwaniami. W testowaniu jednostkowym zastosowano bibliotekę \textbf{pytest} oraz narzędzie \textbf{unittest} w Pythonie.

Przykładowe testy jednostkowe obejmowały:
\begin{itemize}
    \item Sprawdzenie poprawności działania funkcji hashującej hasła użytkowników,
    \item Weryfikację generowania i walidacji tokenów JWT,
    \item Testy funkcji odpowiedzialnych za zarządzanie użytkownikami (dodawanie, edycja, usuwanie),
    \item Weryfikację poprawności operacji CRUD dla bazy danych MongoDB.
\end{itemize}

Wszystkie testy jednostkowe zostały zautomatyzowane i uruchamiane w ramach procesu CI/CD z wykorzystaniem GitHub Actions oraz samodzielnie hostowanych runnerów.

\subsection{Testy integracyjne}

Testy integracyjne mają na celu sprawdzenie współpracy różnych komponentów systemu, takich jak API, baza danych oraz interfejs użytkownika. W ramach pracy przeprowadzono testy integracyjne z użyciem frameworka \textbf{pytest} wraz z modułem \textbf{httpx}, umożliwiającym wysyłanie zapytań HTTP do testowanego serwera.

Zakres testów integracyjnych obejmował:
\begin{itemize}
    \item Sprawdzenie poprawności obsługi żądań API przez FastAPI,
    \item Testy poprawności integracji systemu uwierzytelniania JWT z bazą danych,
    \item Weryfikację działania operacji na usługach, takich jak rejestrowanie, uruchamianie i zatrzymywanie kontenerów.
\end{itemize}

Testy integracyjne zostały przeprowadzone zarówno w środowisku lokalnym, jak i w środowisku testowym z wykorzystaniem kontenerów Docker.

\subsection{Podsumowanie testów}

Przeprowadzone testy jednostkowe i integracyjne pozwoliły na wykrycie oraz eliminację potencjalnych błędów na wczesnym etapie rozwoju aplikacji. Dzięki zastosowaniu automatyzacji testów oraz integracji z procesem CI/CD, system został zoptymalizowany pod kątem stabilności i niezawodności. Testowanie stanowiło istotny element procesu wdrażania i potwierdziło poprawność działania kluczowych funkcjonalności systemu HomeLab.

\section{Testy wydajnościowe i bezpieczeństwa}
\subsection{Testowanie wydajności systemu}

Testowanie wydajności aplikacji jest kluczowym elementem zapewnienia jej stabilności i efektywności w warunkach produkcyjnych. W celu przeprowadzenia testów wydajnościowych wykorzystano narzędzie \textbf{Locust}, które pozwala na symulację obciążenia aplikacji przez wielu użytkowników jednocześnie.

\subsubsection{Cel testów wydajnościowych}
Celem testów wydajnościowych było:
\begin{itemize}
    \item Ocena wydajności API w warunkach wysokiego obciążenia,
    \item Pomiar czasu odpowiedzi kluczowych endpointów,
    \item Weryfikacja stabilności systemu podczas długotrwałego obciążenia,
    \item Identyfikacja potencjalnych wąskich gardeł aplikacji.
\end{itemize}

\subsubsection{Metodyka testów}
Testy wydajnościowe przeprowadzono przy użyciu Locust, który pozwala na definiowanie scenariuszy użytkowników wykonujących określone operacje. Wykorzystano następujące kroki:
\begin{enumerate}
    \item Zalogowanie użytkownika i uzyskanie tokena JWT,
    \item Wykonywanie żądań do kluczowych endpointów API (zarządzanie usługami, monitorowanie systemu, operacje administracyjne),
    \item Pomiar czasu odpowiedzi serwera i obciążenia systemu,
    \item Skalowanie liczby użytkowników w celu oceny zachowania systemu pod rosnącym obciążeniem.
\end{enumerate}

\subsubsection{Zakres testów}
W ramach testów wydajnościowych poddano analizie następujące funkcjonalności:
\begin{itemize}
    \item \textbf{Autoryzacja i uwierzytelnianie} – logowanie użytkowników i uzyskiwanie tokenów JWT,
    \item \textbf{Zarządzanie usługami} – pobieranie listy uruchomionych usług, monitorowanie ich stanu oraz restartowanie usług w złym stanie,
    \item \textbf{Monitorowanie systemu} – pobieranie danych dotyczących zużycia CPU, pamięci RAM oraz listy aktywnych kontenerów Docker,
    \item \textbf{Zarządzanie użytkownikami} – dodawanie i listowanie użytkowników.
\end{itemize}

\subsubsection{Wyniki testów}
Po przeprowadzeniu testów otrzymano następujące wyniki:
\begin{itemize}
    \item Średni czas odpowiedzi API dla standardowych żądań wynosił poniżej 70 ms (lokalne środowisko - w tej samej sieci),
    \item Przy zwiększeniu liczby jednocześnie aktywnych użytkowników do 1000 system nadal utrzymywał stabilność, przy wzroście czasu odpowiedzi do około 190 ms,
    \item Największe obciążenie dotyczyło operacji związanych z monitorowaniem systemu, co wynika z konieczności pobierania danych o stanie zasobów w czasie rzeczywistym - wskazuje to na konieczność zastosowania cachowania danych dotyczących stanu systemu celem uniknięcia ataktu DDoS,
    \item Endpointy administracyjne, takie jak restart serwera czy autoryzacja w Tailscale, działały poprawnie, lecz wymagają dodatkowych zabezpieczeń przed wielokrotnym wywołaniem w krótkim czasie.
\end{itemize}

\subsubsection{Wnioski i optymalizacje}
Na podstawie przeprowadzonych testów wydajnościowych zaproponowano następujące optymalizacje:
\begin{itemize}
    \item Implementacja mechanizmu cache'owania dla danych monitorowania systemu w celu zmniejszenia liczby odczytów zasobów,
    \item Ograniczenie liczby jednoczesnych zapytań do endpointów administracyjnych,
    \item Optymalizacja zapytań do bazy danych poprzez indeksowanie często wyszukiwanych pól.
\end{itemize}

Testy wykazały, że system jest w stanie obsłużyć duże obciążenie i zachowuje stabilność przy wysokiej liczbie równoczesnych użytkowników, co potwierdza jego gotowość do wdrożenia w środowisku produkcyjnym.

\subsection{Testowanie bezpieczeństwa aplikacji}

Bezpieczeństwo aplikacji webowych jest kluczowym aspektem zapewnienia poufności, integralności i dostępności danych. Testowanie bezpieczeństwa ma na celu identyfikację potencjalnych luk oraz podatności, które mogą zostać wykorzystane przez nieautoryzowanych użytkowników lub atakujących.

\textbf{Należy zaznaczyć, że w ramach niniejszej pracy magisterskiej testy bezpieczeństwa nie zostały przeprowadzone. Poniższy rozdział ma charakter teoretyczny i przedstawia ogólne zasady oraz metody stosowane w testowaniu bezpieczeństwa aplikacji webowych.}

\subsubsection{Metodyka testowania bezpieczeństwa}
Testowanie bezpieczeństwa można podzielić na kilka kluczowych etapów:
\begin{itemize}
    \item \textbf{Analiza architektury} – Przegląd struktury aplikacji, mechanizmów autoryzacji i uwierzytelniania,
    \item \textbf{Testy penetracyjne} – Symulowane ataki w celu sprawdzenia odporności na znane zagrożenia,
    \item \textbf{Analiza kodu źródłowego} – Poszukiwanie błędów bezpieczeństwa w implementacji aplikacji,
    \item \textbf{Fuzzing} – Automatyczne generowanie losowych danych wejściowych w celu wykrycia awarii,
    \item \textbf{Skany podatności} – Wykorzystanie narzędzi do identyfikacji znanych luk w zabezpieczeniach.
\end{itemize}

\subsubsection{Obszary testowania bezpieczeństwa}
Testowanie bezpieczeństwa aplikacji obejmuje następujące obszary:

\paragraph{1. Uwierzytelnianie i autoryzacja}
Mechanizmy uwierzytelniania powinny być odporne na ataki brute-force oraz przechowywać hasła w bezpieczny sposób. Kluczowe testy obejmują:
\begin{itemize}
    \item Testowanie siły haseł i polityki logowania,
    \item Próby ataków brute-force na endpointy logowania,
    \item Weryfikacja poprawności implementacji JWT i zarządzania sesjami,
    \item Sprawdzenie uprawnień użytkowników do zasobów.
\end{itemize}

\paragraph{2. Zarządzanie danymi i ochrona przed SQL Injection}
Baza danych powinna być zabezpieczona przed atakami wstrzykiwania SQL. Weryfikacja obejmuje:
\begin{itemize}
    \item Testy podatności na SQL Injection przy użyciu specjalnie spreparowanych zapytań,
    \item Sprawdzenie, czy aplikacja korzysta z mechanizmów ORM i zapytań parametryzowanych,
    \item Ograniczenie uprawnień użytkowników bazy danych.
\end{itemize}

\paragraph{3. Ochrona przed atakami XSS i CSRF}
Ataki Cross-Site Scripting (XSS) i Cross-Site Request Forgery (CSRF) mogą prowadzić do przejęcia sesji użytkownika lub wykonania nieautoryzowanych operacji. Testy obejmują:
\begin{itemize}
    \item Wstrzykiwanie skryptów JavaScript w formularzach i żądaniach API,
    \item Weryfikację nagłówków zabezpieczających (Content Security Policy, SameSite Cookies),
    \item Sprawdzenie zabezpieczeń przed atakami CSRF poprzez implementację tokenów anty-CSRF.
\end{itemize}

\paragraph{4. Bezpieczeństwo API}
Bezpieczeństwo API jest kluczowe dla ochrony danych przesyłanych pomiędzy klientem a serwerem. Testowanie obejmuje:
\begin{itemize}
    \item Sprawdzenie poprawności nagłówków autoryzacyjnych,
    \item Testy na ataki replay (ponowne wykorzystanie żądań API),
    \item Weryfikację mechanizmów rate-limiting (ograniczenie liczby żądań w czasie),
    \item Ochronę przed atakami Man-in-the-Middle (MITM) poprzez wymuszanie HTTPS.
\end{itemize}

\paragraph{5. Odporność na ataki DoS/DDoS}
Ataki Denial of Service (DoS) mogą doprowadzić do przeciążenia aplikacji i uniemożliwienia jej działania. Testy obejmują:
\begin{itemize}
    \item Symulację dużej liczby jednoczesnych żądań,
    \item Weryfikację mechanizmów ograniczających dostępność zasobów,
    \item Sprawdzenie konfiguracji serwera pod kątem ochrony przed atakami DDoS.
\end{itemize}

\subsubsection{Narzędzia do testowania bezpieczeństwa}
Do testowania bezpieczeństwa wykorzystuje się różne narzędzia, m.in.:
\begin{itemize}
    \item \textbf{OWASP ZAP} – Automatyczne skanowanie podatności aplikacji webowych,
    \item \textbf{Burp Suite} – Analiza i przechwytywanie ruchu HTTP/HTTPS,
    \item \textbf{sqlmap} – Automatyczne wykrywanie podatności na SQL Injection,
    \item \textbf{Metasploit} – Wykonywanie testów penetracyjnych,
    \item \textbf{Fail2Ban} – Ochrona przed atakami brute-force poprzez blokowanie adresów IP.
\end{itemize}

\subsubsection{Podsumowanie}
Testowanie bezpieczeństwa aplikacji webowych jest kluczowym elementem zapewnienia ochrony danych i usług. Wykorzystanie kompleksowego podejścia, obejmującego testy penetracyjne, analizę kodu oraz automatyczne skanowanie podatności, pozwala na wczesne wykrycie zagrożeń i skuteczne zabezpieczenie systemu przed atakami. Regularne testy i aktualizacje mechanizmów bezpieczeństwa są niezbędne do utrzymania wysokiego poziomu ochrony aplikacji.

\textbf{Podkreśla się, że powyższe informacje mają charakter teoretyczny i w ramach niniejszej pracy magisterskiej nie przeprowadzono rzeczywistych testów bezpieczeństwa aplikacji.}