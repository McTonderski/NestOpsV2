W ramach tworzenia rozwiązania powstał również program instalacyjny, który zostanie umieszczony na stronie internetowej rozwiązania umożliwiając pobranie przez zainteresowane korzystaniem z rozwiązania osoby.
\paragraph{Program instalacyjny}\mbox{}\\

W celu ułatwienia instalacji aplikacji na systemach użytkowników końcowych przygotowano dedykowany program instalacyjny napisany w języku Go. Aplikacja ta jest odpowiedzialna za:

\begin{itemize}
    \item Weryfikację obecności Dockera w systemie,
    \item Automatyczną instalację Dockera w systemach Linux (w pozostałych przypadkach użytkownik zostaje poinformowany o konieczności ręcznej instalacji),
    \item Pobranie najnowszej wersji aplikacji HomeNest z repozytorium GitHub,
    \item Rozpakowanie archiwum ZIP zawierającego aplikację do wskazanego katalogu,
    \item Nadanie odpowiednich uprawnień do uruchomienia aplikacji oraz jej uruchomienie.
\end{itemize}

Program wykorzystuje standardowe biblioteki Go do obsługi pobierania plików (\texttt{net/http}), rozpakowywania archiwów ZIP (\texttt{archive/zip}) oraz uruchamiania procesów lokalnych (\texttt{os/exec}). Dzięki temu działa w pełni samodzielnie, bez konieczności instalowania zewnętrznych zależności.

Logika działania programu obejmuje:
\begin{enumerate}
    \item Sprawdzenie czy polecenie \texttt{docker} znajduje się w ścieżce systemowej.
    \item Jeżeli Docker nie jest zainstalowany — uruchomienie skryptu instalacyjnego (tylko w systemach Linux).
    \item Pobranie pliku ZIP z GitHub Releases.
    \item Rozpakowanie pliku do katalogu roboczego.
    \item Uruchomienie pliku binarnego aplikacji.
\end{enumerate}

Instalator został zaprojektowany z myślą o prostocie użytkowania oraz możliwości łatwego wdrożenia aplikacji przez osoby nietechniczne. Wersje programu można przygotować na platformy Linux, Windows i macOS z wykorzystaniem narzędzia \texttt{GOOS/GOARCH}, umożliwiającego cross-kompilację.