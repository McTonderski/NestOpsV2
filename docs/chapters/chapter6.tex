\chapter{Podsumowanie i wnioski}

\section{Podsumowanie pracy}

W niniejszej pracy magisterskiej przedstawiono projekt, implementację oraz analizę systemu zarządzania usługami i monitorowania serwera, napisanego w języku Go. System składa się z backendu (API) stworzonego w Go oraz frontendowej części zbudowanej w oparciu o szablon TailAdmin. Celem pracy było stworzenie wydajnej i bezpiecznej aplikacji umożliwiającej administratorom zarządzanie usługami systemowymi, monitorowanie wykorzystania zasobów oraz przeprowadzanie operacji administracyjnych na serwerze.

W trakcie realizacji projektu skupiono się na kilku kluczowych aspektach, w tym implementacji API w Go, integracji z interfejsem użytkownika TailAdmin, zarządzaniu użytkownikami, optymalizacji wydajności oraz aspektach związanych z bezpieczeństwem aplikacji webowych. Każdy z tych elementów został szczegółowo opisany i przeanalizowany, co pozwoliło na wyciągnięcie cennych wniosków dotyczących zarówno języka Go, jak i nowoczesnych narzędzi frontendowych i backendowych stosowanych w zarządzaniu infrastrukturą IT.

Instrukcja instalacji oraz uruchomienia aplikacji została zamieszczona w pliku \texttt{README.md} w repozytorium projektu na platformie GitHub\cite{NestOpsV2}.


\subsection{Osiągnięcia i rezultaty pracy}

W ramach realizacji pracy magisterskiej osiągnięto szereg istotnych rezultatów, które potwierdzają praktyczną użyteczność oraz skalowalność zaprojektowanego systemu. Najważniejsze z nich to:

\begin{itemize}
    \item Zaprojektowano oraz zaimplementowano backend systemu przy użyciu języka Go. Dzięki zastosowaniu tej technologii aplikacja cechuje się wysoką wydajnością, prostotą dystrybucji oraz łatwością utrzymania.
    \item Opracowano nowoczesny, responsywny interfejs użytkownika wykorzystując szablon TailAdmin. Interfejs ten umożliwia intuicyjne zarządzanie usługami systemowymi i użytkownikami, a także prezentuje dane o stanie systemu w przejrzystej formie.
    \item Stworzono kompletne REST API umożliwiające zarządzanie użytkownikami, usługami oraz monitorowanie parametrów systemowych. API zostało zaprojektowane w sposób modularny i rozszerzalny.
    \item Wdrożono mechanizmy uwierzytelniania i autoryzacji użytkowników z wykorzystaniem JWT. System obsługuje różne poziomy uprawnień oraz zabezpiecza dostęp do krytycznych funkcjonalności.
    \item Zaimplementowano funkcjonalności administracyjne obejmujące dodawanie, edytowanie, usuwanie i przeglądanie użytkowników z poziomu panelu administratora.
    \item Umożliwiono rejestrowanie i zarządzanie usługami systemowymi – uruchamianie, zatrzymywanie oraz monitorowanie ich stanu w czasie rzeczywistym.
    \item Zintegrowano backend z systemem operacyjnym w zakresie kontroli nad aktywnymi usługami systemowymi i dostępem do danych o zużyciu zasobów (RAM, CPU, dysk).
    \item Przeprowadzono testy wydajnościowe z użyciem narzędzia Ddosify, które wykazały odporność aplikacji na duże obciążenia oraz pomogły zidentyfikować krytyczne momenty wymagające optymalizacji.
    \item Zrealizowano testy bezpieczeństwa, w tym odporności na ataki typu brute-force, dostęp nieautoryzowany, SQL Injection oraz Cross-Site Scripting, potwierdzając bezpieczeństwo podstawowych mechanizmów aplikacji.
    \item Opracowano system budowania i pakowania aplikacji za pomocą Makefile, umożliwiający łatwą kompilację na wiele platform oraz automatyczne tworzenie paczek instalacyjnych.
    \item Udokumentowano projekt w postaci pliku \texttt{README.md} na GitHubie, w którym zawarto instrukcję instalacji, konfiguracji i uruchomienia systemu zarówno na lokalnych maszynach, jak i w środowiskach produkcyjnych.
\end{itemize}

System powstały w wyniku realizacji pracy magisterskiej został zaprojektowany zgodnie z zasadami czystej architektury i modularności, co pozwala na jego dalsze rozwijanie i dostosowywanie do indywidualnych potrzeb użytkowników oraz specyfiki środowiska produkcyjnego. Aktualna wersja spełnia wszystkie założenia funkcjonalne, stanowiąc stabilną podstawę dla rzeczywistego wdrożenia.

\subsection{Wnioski i przyszłe kierunki rozwoju}

Na podstawie przeprowadzonych badań, testów oraz implementacji można sformułować następujące wnioski:

\begin{itemize}
    \item Połączenie języka Go w warstwie backendu oraz szablonu TailAdmin w warstwie frontendowej pozwala na szybkie tworzenie aplikacji o wysokiej wydajności i przejrzystym interfejsie użytkownika.
    \item Architektura systemu została zaprojektowana w sposób umożliwiający jego łatwą rozbudowę – zarówno poprzez dodawanie nowych endpointów API, jak i integrację z zewnętrznymi usługami.
    \item Testy wydajnościowe ujawniły, że system dobrze radzi sobie z typowym obciążeniem, jednak należy zwrócić uwagę na ograniczenia wynikające z zastosowania SQLite, szczególnie przy dużej liczbie jednoczesnych operacji zapisu.
    \item Mechanizmy bezpieczeństwa wdrożone w systemie (autoryzacja JWT, kontrola ról, odporność na podstawowe ataki webowe) są wystarczające na etapie prototypowania, jednak wymagają dalszej rozbudowy przed wdrożeniem produkcyjnym.
\end{itemize}

W przyszłości system może zostać rozszerzony o:

\begin{itemize}
    \item Wdrożenie kolejki zadań (np. z użyciem Redis lub RabbitMQ) w celu obsługi operacji asynchronicznych oraz przetwarzania zadań wymagających więcej czasu.
    \item Rozszerzenie API o nowe funkcjonalności, takie jak integracja z narzędziami CI/CD, mechanizmy webhooków, czy eksport danych do zewnętrznych systemów.
    \item Zastosowanie bazy danych wspierającej jednoczesne zapisy (np. PostgreSQL) dla zwiększenia skalowalności.
    \item Integrację z narzędziami monitorującymi (np. Prometheus, Grafana) i systemami SIEM w celu zwiększenia widoczności operacyjnej i poziomu bezpieczeństwa.
    \item Wprowadzenie funkcjonalności cache’owania danych, aby zmniejszyć obciążenie backendu i przyspieszyć odpowiedzi aplikacji.
    \item Wdrożenie mechanizmów uwierzytelniania wieloskładnikowego (MFA) oraz integrację z zewnętrznymi systemami tożsamości (np. LDAP, OAuth).
\end{itemize}

Podsumowując, opracowana aplikacja stanowi solidne i nowoczesne rozwiązanie, które może być wykorzystane zarówno w małych środowiskach serwerowych, jak i jako baza pod rozbudowane systemy administracji IT. Projekt może zostać rozwinięty o nowe funkcje i zintegrowany z narzędziami stosowanymi w profesjonalnym zarządzaniu infrastrukturą informatyczną.


\section{Możliwości dalszego rozwoju systemu}

Opracowany system zarządzania usługami i monitorowania serwera został zaprojektowany w sposób modularny i skalowalny, co pozwala na jego dalszą rozbudowę. Możliwe kierunki rozwoju obejmują zarówno ulepszenia funkcjonalne, jak i wdrożenie zaawansowanych mechanizmów zwiększających wydajność oraz bezpieczeństwo systemu. W niniejszym rozdziale przedstawiono propozycje rozszerzeń, które mogą znacząco zwiększyć wartość aplikacji w praktycznym zastosowaniu.

\subsection{Rozszerzenie funkcjonalności zarządzania usługami}
Jednym z kluczowych aspektów rozwoju systemu jest zwiększenie możliwości zarządzania usługami. W obecnej wersji administratorzy mogą rejestrować, uruchamiać, zatrzymywać i monitorować usługi. Można jednak wprowadzić dodatkowe funkcjonalności, takie jak:
\begin{itemize}
    \item \textbf{Automatyczna rekonfiguracja usług} – możliwość dynamicznego dostosowywania parametrów działania usług na podstawie monitorowanych wskaźników wydajności,
    \item \textbf{Harmonogramowanie zadań} – funkcja umożliwiająca administratorom zaplanowanie uruchamiania lub restartowania usług w określonych przedziałach czasowych,
    \item \textbf{Rejestrowanie logów systemowych} – pełna historia zmian w stanie usług wraz z integracją z narzędziami do analizy logów (np. ELK Stack),
    \item \textbf{Automatyczna naprawa błędów} – system wykrywania i samonaprawy usług w przypadku wykrycia awarii.
\end{itemize}

\subsection{Zaawansowane mechanizmy monitorowania}
Obecnie system pozwala na podstawowe monitorowanie wykorzystania CPU, pamięci RAM oraz aktywnych usług. Możliwości dalszego rozwoju obejmują:
\begin{itemize}
    \item \textbf{Monitorowanie wykorzystania sieci} – analiza ruchu sieciowego generowanego przez poszczególne usługi,
    \item \textbf{Alerty i powiadomienia} – implementacja powiadomień o przekroczeniu krytycznych wartości obciążenia systemu (NTFY \cite{NTFY}, Slack \cite{Slack}, Telegram \cite{Telegram}),
    \item \textbf{Predykcja awarii} – zastosowanie algorytmów uczenia maszynowego do przewidywania potencjalnych awarii na podstawie analizy historycznych danych,
    \item \textbf{Dashboard w czasie rzeczywistym} – interaktywna wizualizacja danych systemowych z aktualizacją w czasie rzeczywistym,
    \item \textbf{Integracja z Prometheus i Grafana} – zaawansowane narzędzia monitorowania umożliwiające gromadzenie metryk i ich wizualizację.
\end{itemize}

\subsection{Optymalizacja wydajności systemu}
Testy wydajnościowe wykazały, że system działa sprawnie, ale jego dalszy rozwój może skupić się na:
\begin{itemize}
    \item \textbf{Implementacji cache'owania danych} – redukcja liczby zapytań do bazy danych i API poprzez zastosowanie Redis lub dekoratora cache,
    \item \textbf{Asynchronicznego przetwarzania operacji} – wdrożenie systemu kolejkowania zadań (np. Celery) dla długotrwałych operacji,
    \item \textbf{Load Balancing} – równoważenie obciążenia poprzez podział ruchu między wiele instancji serwera API,
    \item \textbf{Obsługa kontenerów} – rozszerzenie systemu o pełne zarządzanie kontenerami Docker, w tym automatyczne skalowanie usług w zależności od obciążenia.
\end{itemize}

\subsection{Zaawansowane mechanizmy bezpieczeństwa}
Chociaż w pracy omówiono teoretyczne aspekty bezpieczeństwa, możliwe są dodatkowe usprawnienia:
\begin{itemize}
    \item \textbf{Wdrożenie uwierzytelniania wieloskładnikowego (MFA)} – zwiększenie poziomu bezpieczeństwa użytkowników,
    \item \textbf{Rozszerzone uprawnienia użytkowników} – możliwość nadawania niestandardowych ról z precyzyjnie określonymi uprawnieniami,
    \item \textbf{Audyt logów i analiza zachowań} – monitorowanie aktywności użytkowników oraz automatyczne wykrywanie podejrzanych działań,
    \item \textbf{Automatyczne skanowanie podatności} – integracja z narzędziami do analizy bezpieczeństwa, np. OWASP ZAP czy Nessus,
    \item \textbf{Szyfrowanie danych wrażliwych} – wdrożenie szyfrowania kluczowych danych przechowywanych w systemie.
\end{itemize}

\subsection{Integracja z innymi systemami}
W celu zwiększenia użyteczności systemu warto rozważyć jego integrację z innymi rozwiązaniami IT, np.:
\begin{itemize}
    \item \textbf{Integracja z Active Directory / LDAP} – centralne zarządzanie użytkownikami i uprawnieniami,
    \item \textbf{Integracja z systemami DevOps} – połączenie z narzędziami CI/CD (Jenkins, GitHub Actions) w celu automatyzacji wdrożeń,
    \item \textbf{API publiczne} – umożliwienie innym systemom korzystania z funkcjonalności poprzez bezpieczne, udokumentowane API,
    \item \textbf{Obsługa wielu serwerów} – rozbudowa systemu do zarządzania wieloma maszynami w ramach jednej platformy,
    \item \textbf{Integracja z chmurą} – możliwość wdrożenia systemu w chmurach publicznych (AWS, Azure, GCP) i zarządzania zasobami.
\end{itemize}

\subsection{Podsumowanie}
Proponowane kierunki rozwoju pokazują szerokie możliwości dalszej rozbudowy systemu. Obecna wersja stanowi solidną bazę do wprowadzania nowych funkcjonalności, zarówno pod kątem optymalizacji działania, jak i zwiększenia bezpieczeństwa oraz zakresu zastosowań. W przyszłości system może ewoluować w kierunku kompleksowego narzędzia do zarządzania infrastrukturą IT, łącząc aspekty monitorowania, automatyzacji i bezpieczeństwa w jednym rozwiązaniu. Dzięki elastycznej architekturze opartej na API wykonanym w GO GIN oraz szablonu frontendu TailAdmin, aplikacja jest gotowa na integrację z innymi systemami i dalszy rozwój w zależności od potrzeb użytkowników.