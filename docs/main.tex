\documentclass[a4paper,12pt]{report}
\usepackage[utf8]{inputenc}
\usepackage[T1]{fontenc}
\usepackage{geometry}
\geometry{left=2.5cm, right=2.5cm, top=2.5cm, bottom=2.5cm}
\usepackage{graphicx} % For including graphics
\usepackage{setspace} % For line spacing
\usepackage{hyperref} % For clickable references
\usepackage{titlesec} % For custom section titles
\usepackage{tocloft} % For customizing the table of contents
\usepackage{biblatex} % For bibliography management
\usepackage{listing}
\addbibresource{bibliography.bib}

% Include additional settings
% Edytorskie wymagania
\usepackage{times} % Times New Roman
\usepackage{titlesec}
\usepackage{fancyhdr}
\usepackage{etoolbox}

% Czcionka bazowa 12pt, justowanie i wcięcie akapitów
\renewcommand{\familydefault}{\rmdefault}
\setlength{\parindent}{1.25cm}
\setlength{\parskip}{0pt}
\linespread{1.5}

% Marginesy
\PassOptionsToPackage{a4paper, top=2.5cm, bottom=2.5cm, left=2.5cm, right=2.5cm}{geometry}

% Tytuły rozdziałów, podrozdziałów
\titleformat{\chapter}[hang]{\bfseries\LARGE}{\thechapter.}{1em}{}
\titlespacing*{\chapter}{0pt}{*0}{12pt}

\titleformat{\section}[hang]{\bfseries\Large}{\thesection}{1em}{}
\titlespacing*{\section}{0pt}{*0}{12pt}

\titleformat{\subsection}[hang]{\bfseries\normalsize}{\thesubsection}{1em}{}
\titlespacing*{\subsection}{0pt}{*0}{12pt}

% Numeracja stron w stopce, wyśrodkowana
\pagestyle{fancy}
\fancyhf{}
\fancyfoot[C]{\thepage}
\renewcommand{\headrulewidth}{0pt}
\renewcommand{\footrulewidth}{0pt}

% Nowa strona dla każdego rozdziału i części
\usepackage{titlesec}
% \newcommand{\sectionbreak}{\clearpage}
\preto\chapter{\clearpage}

\usepackage{geometry}
\usepackage[polish]{babel}
% Hyperlink setup
\hypersetup{
    colorlinks=true,
    linkcolor=blue,
    citecolor=red,
    urlcolor=blue
}

% Table of Contents customization
\renewcommand{\cftchapfont}{\bfseries}
\renewcommand{\cftsecfont}{\normalfont}
\renewcommand{\cftsubsecfont}{\normalfont}
\renewcommand*{\figurename}{Rys.}
\renewcommand*{\tablename}{Tab.}
\setlength{\cftbeforechapskip}{1em}

% Packages
\usepackage{listings}
\usepackage{xcolor}
\lstdefinelanguage{Go}{
  morekeywords={break,case,chan,const,continue,default,defer,else,fallthrough,
    for,func,go,goto,if,import,interface,map,package,range,return,select,
    struct,switch,type,var},
  sensitive=true,
  morecomment=[l]{//},
  morecomment=[s]{/*}{*/},
  morestring=[b]",
}
\lstset{
  language=Go,
  basicstyle=\ttfamily\footnotesize,
  keywordstyle=\color{blue},
  commentstyle=\color{gray},
  stringstyle=\color{teal},
  showstringspaces=false,
  breaklines=true,
  frame=single,
  tabsize=2,
  captionpos=b
}
\usepackage{courier} % czcionka o stałej szerokości
\lstset{
  basicstyle=\footnotesize\ttfamily,
  breaklines=true,
  frame=single,
  columns=fullflexible
}

% images to stay in place text not to appear in between 
\usepackage{float}

\begin{document}

% Title page
\begin{titlepage}
    \begin{center}
    
    % University Logo
    \includegraphics[width=1\textwidth]{images/logo.jpg} % Replace with your university's logo file
    
    \vspace{1cm}
    
    
    % Faculty, Department, Major, and Specialization
    \textbf{\large WYDZIAŁ} \\
    \textbf{\large BUDOWY MASZYN I INFORMATYKI} \\
    
    \vspace{0.3cm}
    
    \textbf{\large KIERUNEK:} \textbf{Informatyka} \\
    
    \vspace{0.3cm}
    
    \textbf{\large SPECJALNOŚĆ:} \textbf{TECHNIKI TWORZENIA OPROGRAMOWANIA}
    
    \vspace{2cm}
    
    % Author and Thesis Information
    \textbf{\Large Maciej Tonderski} \\
    
    \vspace{0.5cm}
    
    \textbf{nr albumu:  62572} \\
    
    \vspace{0.5cm}
    
    \textbf{\large Praca magisterska}
    
    \vspace{1.5cm}
    
    % Title of the Thesis
    \textbf{\LARGE Uproszczenie procesu wdrażania środowisk HomeLab poprzez projekt i implementację zintegrowanego systemu zarządzania.} \\
    
    \vspace{0.5cm}
    
    \textit{Kategoria pracy: projektowa}
    
    \vspace{2cm}
    
    % Supervisor and Advisor
    \begin{flushright}
    Promotor: dr inż. RUSLAN SHEVCHUK \\
    
    \vspace{0.5cm}
    \end{flushright}
    
    \vfill
    
    % Location and Date
    Bielsko-Biała, 2025 \\
    
    \end{center}
    \end{titlepage}

% Abstract
\section*{Streszczenie}

W niniejszej pracy magisterskiej przedstawiono projekt i implementację systemu HomeLab, który umożliwia użytkownikom łatwe zarządzanie infrastrukturą IT w środowisku domowym. Celem projektu było stworzenie rozwiązania, które pozwala na automatyczne wdrażanie i kontrolowanie maszyn wirtualnych, kontenerów oraz zasobów sieciowych, przy jednoczesnym zapewnieniu wysokiego poziomu bezpieczeństwa i intuicyjności obsługi.

Praca rozpoczyna się od omówienia koncepcji HomeLab oraz analizy istniejących rozwiązań do zarządzania infrastrukturą IT, takich jak Proxmox, Unraid, Docker, Kubernetes i inne narzędzia do automatyzacji oraz monitorowania. Wskazano ich zalety i ograniczenia, co pozwoliło na określenie luki technologicznej, którą uzupełnia proponowany system.

Kolejne rozdziały opisują szczegóły techniczne dotyczące architektury systemu, sposobu implementacji oraz wykorzystanych technologii. Główne elementy składające się na rozwiązanie to:
\begin{itemize}
    \item Backend stworzony w technologii GoLang zarządzający uzytkownikami, usługami oraz monitorujący system.
    \item Baza danych SQLite przechowująca informację o zarejestrowanych uzytkownikach, przeprowadzonych ostatnich pomiarach oraz monitorująca
    \item Interfejs użytkownika stworzony przy użyciu TypeScript oraz szablonu TailAdmin \cite{TailAdmin},
    \item Mechanizmy uwierzytelniania i autoryzacji z wykorzystaniem JWT, zapewniające bezpieczeństwo operacji,
\end{itemize}

W pracy przeprowadzono również testy wydajnościowe, które potwierdziły, że system jest w stanie obsłużyć duże obciążenie i działa stabilnie przy wysokiej liczbie jednoczesnych użytkowników. Ponadto przedstawiono teoretyczne aspekty testowania bezpieczeństwa, wskazując najlepsze praktyki oraz narzędzia stosowane do analizy podatności systemu.

Opracowany system HomeLab stanowi kompleksowe i skalowalne rozwiązanie, które może być wykorzystywane przez administratorów IT, pasjonatów nowych technologii oraz osoby chcące zwiększyć swoją kontrolę nad infrastrukturą IT w środowisku domowym. Praca ta dostarcza także szczegółowej analizy istniejących narzędzi oraz wdrożonych mechanizmów, stanowiąc podstawę do dalszego rozwoju podobnych systemów.


\newpage

% Table of Contents
\renewcommand{\contentsname}{Spis treści}

\setcounter{tocdepth}{1}
\tableofcontents
\newpage

% Chapters
\chapter{Wprowadzenie}

Wraz z dynamicznym rozwojem technologii informatycznych, obserwujemy coraz większe zainteresowanie samodzielnym tworzeniem oraz utrzymywaniem środowisk serwerowych poza scentralizowaną infrastrukturą chmurową. Zjawisko to przybiera szczególne znaczenie wśród pasjonatów IT, administratorów systemów, a także inżynierów oprogramowania, którzy decydują się na budowę tzw. \textit{HomeLabów} – domowych laboratoriów IT. 

HomeLab to prywatne środowisko serwerowe, skonfigurowane najczęściej w warunkach domowych, które umożliwia użytkownikom testowanie, uruchamianie i rozwijanie różnorodnych usług oraz technologii. Może ono przybierać formę jednego serwera z kilkoma kontenerami, klastra urządzeń Raspberry Pi, a nawet rozbudowanego racka z profesjonalnymi serwerami. Niezależnie od skali, HomeLab spełnia istotną rolę edukacyjną, testową, a także produkcyjną w kontekście usług dostępnych lokalnie lub zdalnie poprzez sieć prywatną lub publiczną.

Jednym z głównych powodów, dla których użytkownicy decydują się na stworzenie HomeLaba, są potrzeby edukacyjne i chęć zdobycia doświadczenia z technologiami wykorzystywanymi w środowiskach korporacyjnych. HomeLab stanowi bezpieczne środowisko, w którym można bez ryzyka testować nowe narzędzia, technologie oraz scenariusze awaryjne. Dla wielu użytkowników jest to również sposób na centralizację usług domowych – takich jak serwery multimedialne, automatyczne kopie zapasowe, monitoring, systemy automatyki domowej czy własne rozwiązania chmurowe (tzw. \textit{self-hosting}).

Współczesne HomeLaby korzystają z szerokiej gamy technologii – od wirtualizacji (np. Proxmox, VMware, Hyper-V), przez konteneryzację (Docker, Podman), aż po automatyzację z wykorzystaniem narzędzi takich jak Ansible, Terraform czy Packer. Pojawienie się lekkich systemów operacyjnych, niskonapięciowych jednostek obliczeniowych oraz otwartoźródłowych rozwiązań zarządzających umożliwiło rozwój wydajnych i energooszczędnych środowisk domowych.

Mimo wielu zalet, wdrożenie i zarządzanie HomeLabem nie jest zadaniem trywialnym. Konieczność konfiguracji sieci, wirtualnych maszyn, kontenerów, bezpieczeństwa czy uwierzytelnienia użytkowników może być wyzwaniem, szczególnie dla osób stawiających pierwsze kroki w świecie infrastruktury IT. Dodatkowo, manualna administracja systemem bywa czasochłonna i podatna na błędy, co może negatywnie wpłynąć na stabilność działania usług oraz doświadczenie użytkownika.

W tym kontekście kluczową rolę odgrywa automatyzacja. Dzięki niej możliwe jest ograniczenie liczby powtarzalnych czynności administracyjnych, przyspieszenie wdrożeń oraz zminimalizowanie ryzyka błędów konfiguracyjnych. Automatyzacja umożliwia również realizację bardziej zaawansowanych scenariuszy, takich jak:
\begin{itemize}
    \item samonaprawiające się klastry,
    \item dynamiczne skalowanie zasobów,
    \item automatyczne aktualizacje i testy regresji,
    \item ciągła integracja i dostarczanie (CI/CD),
    \item monitorowanie i alertowanie.
\end{itemize}

Dzięki automatyzacji, HomeLab może stać się nie tylko narzędziem edukacyjnym, ale także realnym środowiskiem produkcyjnym obsługującym usługi użytkownika w sposób niezawodny, elastyczny i bezpieczny.

\section{Cel pracy}

Celem niniejszej pracy magisterskiej jest zaprojektowanie i implementacja nowoczesnego systemu zarządzania środowiskiem HomeLab, który będzie wspierał użytkownika w procesie budowy, konfiguracji i obsługi infrastruktury IT w sposób zautomatyzowany, intuicyjny oraz skalowalny. 

Proponowane rozwiązanie ma za zadanie:
\begin{itemize}
    \item Ułatwić wdrażanie i zarządzanie usługami w kontenerach,
    \item Zapewnić dostęp do intuicyjnego interfejsu użytkownika umożliwiającego kontrolę nad całym środowiskiem,
    \item Zminimalizować potrzebę ręcznej ingerencji w konfigurację systemów,
    \item Wspierać integrację z popularnymi rozwiązaniami open-source (np. Docker, SQLite, TailScale),
    \item Zwiększyć bezpieczeństwo i kontrolę nad uruchamianymi usługami.
\end{itemize}

System ma na celu obniżenie progu wejścia dla osób rozpoczynających pracę z HomeLabem oraz dostarczenie bardziej zaawansowanym użytkownikom elastycznej i rozszerzalnej platformy do dalszego rozwoju. Całość zostanie udostępniona jako projekt open-source, co umożliwi społeczności jego dalsze rozwijanie i dostosowywanie.

\section{Zakres pracy}

W ramach niniejszej pracy zostaną omówione następujące zagadnienia:
\begin{itemize}
    \item Analiza dostępnych technologii oraz przegląd istniejących rozwiązań open-source w zakresie zarządzania infrastrukturą domową,
    \item Projektowanie architektury systemu, obejmującej backend, frontend oraz warstwę automatyzującą,
    \item Implementacja interfejsu użytkownika umożliwiającego zdalne zarządzanie HomeLabem,
    \item Budowa API służącego do komunikacji z systemem operacyjnym i usługami backendowymi,
    \item Integracja z kontenerami Docker oraz obsługa uruchamiania, zatrzymywania i monitorowania usług,
    \item Wdrożenie mechanizmów bezpieczeństwa: autoryzacji, uwierzytelnienia i ochrony dostępu,
    \item Przeprowadzenie testów funkcjonalnych oraz wydajnościowych na urządzeniu Raspberry Pi 5,
    \item Udokumentowanie i przygotowanie instalatora umożliwiającego łatwe wdrożenie systemu przez użytkownika końcowego.
\end{itemize}

Opracowane rozwiązanie nie tylko upraszcza proces zarządzania domową infrastrukturą IT, ale stanowi również punkt wyjścia do dalszej rozbudowy. W kolejnych rozdziałach omówione zostaną szczegółowo zarówno decyzje projektowe, jak i konkretne aspekty implementacyjne oraz propozycje dalszego rozwoju systemu. Projekt publikowany jest w repozytorium GitHub, co umożliwia społeczności swobodne korzystanie oraz modyfikowanie aplikacji zgodnie z własnymi potrzebami.

\chapter{Czym jest HomeLab oraz analiza istniejących rozwiązań}

\section{Definicja HomeLab oraz znaczenie}
HomeLab jest prywatnym środowiskiem IT, dzięki któremu entuzjaści nowych technologii, administratorzy systemów oraz programiści mogą w lokalnym – domowym środowisku testować, rozwijać oraz zarządzać własną infrastrukturą IT. Jego głównym zamierzeniem jest stworzenie realistycznego środowiska do eksperymentowania z technologiami chmurowymi, wirtualizacją, konteneryzacją oraz narzędziami DevOps. Własny system HomeLab to również metoda na rezygnację z komercyjnych subskrypcji, takich jak Google Drive, Dropbox czy OneDrive, co pozwala na pełną kontrolę nad dostępem do prywatnych danych. Dzięki niemu zwiększa się prywatność poprzez wyeliminowanie potrzeby przechowywania zdjęć w usługach chmurowych, takich jak Google Photos.

HomeLaby znajdują zastosowanie w wielu obszarach, w tym:
\begin{itemize}
    \item nauka administracji serwerami i sieciami,
    \item testowanie nowych technologii przed użyciem ich w środowisku produkcyjnym,
    \item budowanie prywatnej chmury oraz rozwiązań do przechowywania danych,
    \item analiza bezpieczeństwa i przeprowadzanie testów penetracyjnych,
    \item tworzenie automatyzacji dla infrastruktury IT,
    \item uniezależnienie się od komercyjnych dostawców chmury w celu zwiększenia kontroli nad własnymi danymi.
\end{itemize}

\subsection{Ewolucja HomeLabów}
Historia HomeLabów sięga czasów, gdy entuzjaści IT zaczynali od prostych zestawów serwerów fizycznych w domach, często z pojedynczymi maszynami do nauki i testów. Początkowo były to głównie serwery oparte na systemach Linux lub Windows Server, służące do eksperymentów z sieciami, usługami plików czy prostymi aplikacjami. Wraz z rozwojem technologii wirtualizacji, takich jak VMware czy Hyper-V, HomeLaby zaczęły ewoluować, umożliwiając uruchamianie wielu maszyn wirtualnych na jednym fizycznym serwerze, co znacznie zwiększyło możliwości testowe i oszczędność zasobów.

Kolejnym etapem było pojawienie się konteneryzacji, przede wszystkim dzięki Dockerowi, która pozwoliła na lekkie i szybkie uruchamianie aplikacji w izolowanych środowiskach.

Docker został zaprezentowany w 2013 roku jako projekt firmy dotCloud, która później zmieniła nazwę na Docker Inc. Technologia ta wywodzi się z systemu konteneryzacji opierającego się na mechanizmach LXC (Linux Containers), które były dostępne w jądrze Linuxa już wcześniej, ale trudne w użyciu. Docker wprowadził prosty interfejs do tworzenia, uruchamiania i zarządzania kontenerami, co zrewolucjonizowało sposób dystrybucji i uruchamiania aplikacji. W kolejnych latach Docker zyskał ogromną popularność, stając się de facto standardem w konteneryzacji, a jego rozwój obejmował m.in. wprowadzenie Docker Swarm (2015) – własnego systemu orkiestracji, integrację z chmurami publicznymi oraz rozwój narzędzi do zarządzania cyklem życia kontenerów. W 2017 roku Docker przekazał część swojej technologii do fundacji CNCF w postaci silnika kontenerowego containerd, a sam projekt coraz częściej był wykorzystywany w połączeniu z Kubernetesem, który stał się dominującym rozwiązaniem orkiestracyjnym.

Wprowadzenie orkiestratorów kontenerów, takich jak Kubernetes, umożliwiło zarządzanie złożonymi systemami rozproszonymi nawet w domowym środowisku. Obecnie HomeLaby coraz częściej korzystają z podejścia cloud-native, integrując narzędzia infrastruktury jako kod (IaC), takie jak Terraform czy Ansible, oraz praktyki GitOps, gdzie konfiguracje i wdrożenia są zarządzane poprzez repozytoria Git, co zwiększa powtarzalność i automatyzację.

Ta ewolucja spowodowała, że dzisiejsze HomeLaby to nie tylko miejsca do nauki, ale pełnoprawne środowiska produkcyjne, które mogą obsługiwać usługi multimedialne, automatyzację domową, a nawet prywatne chmury danych i aplikacji.

\section{Technologie wykorzystywane w HomeLabach}

W niniejszej sekcji przedstawiono wybrane technologie i rozwiązania wykorzystywane w środowiskach typu HomeLab. Należy jednak zaznaczyć, że poniższy przegląd ma charakter poglądowy i służy jedynie przedstawieniu gotowych rozwiązań dostępnych na rynku. Opisane narzędzia i systemy mają na celu pogłębienie wiedzy czytelnika w zakresie możliwości technologicznych, ale w większości nie zostały wykorzystane bezpośrednio w implementacji systemu zaprojektowanego w ramach niniejszej pracy.

HomeLab może składać się z różnych komponentów, od dedykowanych serwerów fizycznych po rozwiązania chmurowe i kontenerowe. Kluczowe technologie wykorzystywane w HomeLabach obejmują:

\subsection{Wirtualizacja i konteneryzacja}
\begin{itemize}
    \item \textbf{Proxmox VE} - platforma do zarządzania maszynami wirtualnymi i kontenerami.
    
    Przykład tworzenia maszyny wirtualnej za pomocą CLI Proxmox:
    \begin{verbatim}
    qm create 100 --memory 2048 --net0 virtio,bridge=vmbr0 --cdrom /var/lib/vz/template/iso/ubuntu.iso
    qm start 100
    \end{verbatim}

    \item \textbf{VMware ESXi} - profesjonalne narzędzie do wirtualizacji serwerów.
    
    Przykład automatycznego wdrożenia VM z wykorzystaniem PowerCLI:
    \begin{verbatim}
    New-VM -Name "TestVM" -ResourcePool "Resources" -Datastore "Datastore1" -MemoryGB 4 -NumCPU 2
    \end{verbatim}

    \item \textbf{Hyper-V} - narzędzie do wirtualizacji dostarczane przez Microsoft wraz z systemem Windows.
    
    Przykład tworzenia maszyny wirtualnej PowerShell:
    \begin{verbatim}
    New-VM -Name "LabVM" -MemoryStartupBytes 2GB -Generation 2 -NewVHDPath "C:\VMs\LabVM.vhdx" -NewVHDSizeBytes 40GB
    \end{verbatim}

    \item \textbf{Docker i Kubernetes} - technologie konteneryzacji, pozwalające na elastyczne zarządzanie aplikacjami i zasobami.
    
    Przykład prostego pliku \texttt{docker-compose.yaml}:
    \begin{verbatim}
    version: '3'
    services:
      web:
        image: nginx:latest
        ports:
          - "80:80"
    \end{verbatim}
    
    Przykład uruchomienia playbooka Kubernetes:
    \begin{verbatim}
    kubectl apply -f deployment.yaml
    \end{verbatim}
\end{itemize}

\subsection{Automatyzacja i zarządzanie konfiguracją}
\begin{itemize}
    \item \textbf{Ansible, Terraform, Puppet, Chef} – narzędzia do automatyzacji wdrażania i zarządzania infrastrukturą.

    Przykład użycia Ansible do instalacji Nginx:
    \begin{verbatim}
    - hosts: homelab_servers
      become: yes
      tasks:
        - name: Install Nginx
          apt:
            name: nginx
            state: present
    \end{verbatim}

    Przykład Terraform do stworzenia maszyny wirtualnej na Proxmox:
    \begin{verbatim}
    resource "proxmox_vm_qemu" "vm1" {
      name   = "homelab-vm"
      memory = 2048
      cores  = 2
      disk {
        size = "20G"
      }
    }
    \end{verbatim}
\end{itemize}

\subsection{Monitoring i analiza}
\begin{itemize}
    \item \textbf{Prometheus i Grafana} – rozwiązania do monitorowania wydajności i wizualizacji danych.
    
    Przykład konfiguracji Prometheus scrape config:
    \begin{verbatim}
    scrape_configs:
      - job_name: 'node_exporter'
        static_configs:
          - targets: ['localhost:9100']
    \end{verbatim}
    
    \item \textbf{Zabbix} – platforma do monitorowania infrastruktury IT.
    
    Przykład prostego szablonu Zabbix do monitorowania serwera:
    \begin{verbatim}
    Hostname: homelab-server
    Items:
      - CPU load
      - Memory usage
      - Disk space
    \end{verbatim}
\end{itemize}

\subsection{Bezpieczeństwo i ochrona danych}
Bezpieczeństwo jest kluczowym aspektem każdego HomeLaba, zwłaszcza gdy przechowywane są dane prywatne. Wśród najważniejszych technologii i praktyk znajdują się:

\begin{itemize}
    \item \textbf{VPN} – umożliwia bezpieczne połączenie z domową siecią z dowolnego miejsca. Popularne rozwiązania to OpenVPN, WireGuard.
    
    Przykład podstawowej konfiguracji WireGuard:
    \begin{verbatim}
    [Interface]
    PrivateKey = <private_key>
    Address = 10.0.0.1/24

    [Peer]
    PublicKey = <peer_public_key>
    AllowedIPs = 10.0.0.2/32
    \end{verbatim}

    \item \textbf{Zapory sieciowe (firewalle)} – kontrolują ruch sieciowy, chroniąc przed nieautoryzowanym dostępem. Można używać iptables, ufw lub dedykowanych urządzeń.

    Przykład reguły ufw zezwalającej na ruch SSH:
    \begin{verbatim}
    sudo ufw allow ssh
    sudo ufw enable
    \end{verbatim}

    \item \textbf{Strategie backupu} – regularne tworzenie kopii zapasowych danych i konfiguracji, np. za pomocą rsync, BorgBackup, czy dedykowanych narzędzi chmurowych.

    Przykład backupu katalogu za pomocą rsync:
    \begin{verbatim}
    rsync -av --delete /home/user/data /mnt/backup/
    \end{verbatim}

    \item \textbf{Szyfrowanie danych} – zarówno na poziomie dysków (LUKS, BitLocker), jak i podczas transmisji (TLS).

    Przykład szyfrowania dysku z LUKS:
    \begin{verbatim}
    cryptsetup luksFormat /dev/sdX
    cryptsetup open /dev/sdX encrypted_disk
    \end{verbatim}
\end{itemize}

\section{Analiza istniejących systemów do zarządzania homelabem}
\subsection{Przegląd dostępnych rozwiązań}
Na rynku istnieje kilka systemów umożliwiających zarządzanie homelabem. Do najpopularniejszych należą:
\begin{itemize}
    \item Proxmox VE \cite{Proxmox} - rozbudowane, open-source rozwiązanie do zarządzania maszynami wirtualnymi i kontenerami, oferujące integrację z Ceph i wysoką dostępność.
    \item Unraid \cite{Unraid} - popularne rozwiązanie NAS z obsługą wirtualizacji i kontenerów, cenione za łatwość obsługi ale ograniczone zastosowanie korporacyjne.
    \item OpenStack \cite{OpenStack} - potężna platforma chmurowa, która może być używana do zarządzania homelabem, ale jej skomplikowana konfiguracja sprawia, że nie jest przyjazna dla początkujących użytkowników.
    \item TrueNAS \cite{TrueNAS} - rozbudowane oprogramowanie do zarządzania przestrzenią dyskową, które umożliwia tworzenie prywatnych chmur danych.
    \item Docker \cite{Docker} + Kubernetes \cite{Kubernetes} - stosowane w bardziej zaawansowanych wdrożeniach do zarządzania kontenerami, ale wymagające większej wiedzy technicznej.
    \item CasaOS \cite{CasaOS} - nowoczesny, łatwy w obsłudze system do zarządzania usługami domowymi i multimedialnymi, skierowany do użytkowników z mniejszym doświadczeniem technicznym.
\end{itemize}

\subsection{Zalety i ograniczenia konkurencyjnych systemów}

\subsubsection{Proxmox VE \cite{Proxmox}}
\begin{minipage}{0.45\textwidth}
    Zalety
    \begin{itemize}
        \item Darmowa wersja open-source.
        \item Wsparcie dla maszyn wirtualnych (KVM) i kontenerów (LXC).
        \item Możliwość tworzenia klastrów wysokiej dostępności.
        \item Przykład zastosowania: idealny do środowisk testowych oraz małych firm wymagających elastycznej wirtualizacji.
    \end{itemize}
\end{minipage}\hfil
\begin{minipage}{0.45\textwidth}
    Wady
    \begin{itemize}
        \item Brak pełnej automatyzacji wdrożeń.
        \item Stosunkowo wysoki próg wejścia dla początkujących użytkowników.
        \item Przykład ograniczenia: mniej przyjazny dla użytkowników nieznających CLI lub skryptów.
    \end{itemize}
\end{minipage}

\subsubsection{Unraid \cite{Unraid}}
\begin{minipage}{0.45\textwidth}
    Zalety
    \begin{itemize}
        \item Intuicyjny interfejs użytkownika.
        \item Łatwa obsługa pamięci masowej i kontrolerów.
        \item Przykład zastosowania: idealne do mediów domowych i prostych rozwiązań NAS.
    \end{itemize}
\end{minipage}\hfil
\begin{minipage}{0.45\textwidth}
    Wady
    \begin{itemize}
        \item Model licencyjny oparty na opłacie jednorazowej.
        \item Ograniczona integracja z systemami chmurowymi.
        \item Przykład ograniczenia: mniej elastyczny w środowiskach korporacyjnych.
    \end{itemize}
\end{minipage}

\subsubsection{OpenStack \cite{OpenStack}}
\begin{minipage}{0.45\textwidth}
    Zalety
    \begin{itemize}
        \item Zaawansowane funkcje chmurowe.
        \item Skalowalność i modularność.
        \item Przykład zastosowania: odpowiedni dla dużych środowisk testowych i produkcyjnych.
    \end{itemize}
\end{minipage}\hfil
\begin{minipage}{0.45\textwidth}
    Wady
    \begin{itemize}
        \item Bardzo wysoka trudność wdrożenia.
        \item Wymaga dużej ilości zasobów sprzętowych.
        \item Przykład ograniczenia: niepraktyczny dla małych i średnich HomeLabów.
    \end{itemize}
\end{minipage}

\subsubsection{TrueNAS \cite{TrueNAS}}
\begin{minipage}{0.45\textwidth}
    Zalety
    \begin{itemize}
        \item Silne wsparcie dla przechowywania danych.
        \item Wbudowana replikacja i ochrona RAID.
        \item Przykład zastosowania: idealny do przechowywania dużych zbiorów danych i backupów.
    \end{itemize}
\end{minipage}\hfil
\begin{minipage}{0.45\textwidth}
    Wady
    \begin{itemize}
        \item Skupione głównie na funkcjach NAS.
        \item Brak natywnego wsparcia dla maszyn wirtualnych.
        \item Przykład ograniczenia: ograniczona funkcjonalność w zakresie wirtualizacji.
    \end{itemize}
\end{minipage}

\subsubsection{Docker \cite{Docker} + Kubernetes \cite{Kubernetes}}
\begin{minipage}{0.45\textwidth}
    Zalety
    \begin{itemize}
        \item Elastyczność w zarządzaniu aplikacjami kontenerowymi.
        \item Łatwe skalowanie infrastruktury.
        \item Przykład zastosowania: idealne dla deweloperów aplikacji mikroserwisowych.
    \end{itemize}
\end{minipage}\hfil
\begin{minipage}{0.45\textwidth}
    Wady
    \begin{itemize}
        \item Wymaga dużej wiedzy technicznej.
        \item Brak wsparcia dla maszyn wirtualnych.
        \item Przykład ograniczenia: nieodpowiednie dla użytkowników początkujących.
    \end{itemize}
\end{minipage}

\subsubsection{CasaOS \cite{CasaOS}}
\begin{minipage}{0.45\textwidth}
    Zalety
    \begin{itemize}
        \item Bardzo prosty, intuicyjny interfejs użytkownika.
        \item Łatwa instalacja i konfiguracja.
        \item Skupienie na usługach multimedialnych i domowych.
        \item Przykład zastosowania: idealne dla użytkowników domowych chcących zarządzać multimediami i prostymi usługami.
    \end{itemize}
\end{minipage}\hfil
\begin{minipage}{0.45\textwidth}
    Wady
    \begin{itemize}
        \item Ograniczona skalowalność i funkcjonalność w porównaniu do rozwiązań enterprise.
        \item Brak zaawansowanych funkcji automatyzacji i integracji.
        \item Przykład ograniczenia: nie nadaje się do rozbudowanych środowisk IT.
    \end{itemize}
\end{minipage}

\subsection{Identyfikacja luki technologicznej}
Analiza powyższego porównania dostępnych systemów pokazuje, że żadne z obecnych rozwiązań nie zapewnia jednocześnie:
\begin{itemize}
    \item Pełnej integracji zarządzania maszynami wirtualnymi, kontenerami i przestrzenią dyskową w jednym ekosystemie.
    \item Prostego i intuicyjnego interfejsu dla użytkowników niebędących ekspertami w zarządzaniu infrastrukturą IT.
    \item Natychmiastowej automatyzacji wdrażania, bez konieczności skomplikowanej konfiguracji narzędzi DevOps.
    \item Wbudowanej funkcjonalności związanej z bezpieczeństwem i prywatnością, eliminującej konieczność korzystania z komercyjnych rozwiązań chmurowych.
\end{itemize}

Proponowany system HomeLab ma na celu uzupełnienie tej luki poprzez stworzenie intuicyjnego narzędzia do zarządzania domową infrastrukturą IT, które zapewni łatwość obsługi, pełną automatyzację oraz zwiększoną prywatność użytkowników.

\subsection{Podsumowanie rozdziału}
W niniejszym rozdziale przedstawiono definicję HomeLaba, jego znaczenie oraz ewolucję od prostych serwerów fizycznych do nowoczesnych, zautomatyzowanych i cloud-native środowisk. Omówiono kluczowe technologie wykorzystywane w HomeLabach, w tym wirtualizację, konteneryzację, automatyzację, monitoring oraz aspekty bezpieczeństwa. Analiza istniejących systemów do zarządzania HomeLabem wykazała, że chociaż dostępne rozwiązania oferują szeroki zakres funkcji, żadne z nich nie łączy kompleksowo prostoty obsługi, pełnej integracji oraz automatyzacji, jednocześnie dbając o bezpieczeństwo i prywatność użytkowników.

W związku z tym istnieje wyraźna luka technologiczna, którą proponowany system HomeLab ma wypełnić. Nowa platforma ma za zadanie dostarczyć użytkownikom narzędzie łatwe w obsłudze, pozwalające na kompleksowe zarządzanie infrastrukturą – od maszyn wirtualnych, przez kontenery, po przestrzeń dyskową – z automatyzacją wdrożeń i wbudowanymi mechanizmami bezpieczeństwa. Takie podejście umożliwi zarówno początkującym, jak i zaawansowanym użytkownikom efektywne korzystanie z domowej infrastruktury IT, zwiększając jednocześnie kontrolę nad danymi i minimalizując ryzyko związane z ich przechowywaniem w chmurze komercyjnej.

\chapter{Projekt Systemy Homelab}

\section{Wymagania funkcjonalne i niefunkcjonalne}

\subsection{Wymagania funkcjonalne}

\begin{enumerate}
    \item \textbf{Zarządzanie infrastrukturą} - mozliwość konfiguracji i zarządzania maszynami wierualnymi oraz kontenerami Docker.
    \item \textbf{Panel administracyjny} - intuicyjny interfejs uzytkownika stworzony przy pomocy systemu AppSmith, do zarządzania zasobami systemu.
    \item \textbf{Baza danych} - przechowywanie informacji o konfiguracji systemu i uzytkownikach w MongoDB.
    \item \textbf{Bezpieczny dostęp zdalny} - integracja z Tailscale umozliwaiajca dostęp z dowolonego miejsca na ziemii.
    \item \textbf{Automatyzacja wdrozeń} - wsparcia dla CI/CD za pomocą GitHub Pipelines.
    \item \textbf{Obsługa domeny} - integracja z DuckDNS w celu dynamicznego zarządzania domeną.
    \item \textbf{Monitorowanie zasobów} - mechanizmy zbierania infromacji o wykorzystaniu CPU, pamięci RAM oraz przestrzeni dyskowej.
    \item \textbf{Wsparcie dla rozszerzeń} - mozliwość dodawania nowych funkcji poprzez kontenery Dockera.
    \item \textbf{Łatwe wdrazanie aplikacji} - opcja uruchamiania własnych usług w kontenerach bez konieczności zaawansowanej konfiguracji.
    \item \textbf{Bezpieczne uwierzytelnianie i automatyzacja} - mechanizm logowania oparty na OAuth2 i zarządzanie rolami uzytkowników.
\end{enumerate}

\pagebreak

\subsection{Wymagania niefunkcjonalne}

\begin{enumerate}
    \item \textbf{Niski pobór energii} - system wdrazany na Raspberry Pi 5, co zapeni efektywność energetyczną.
    \item \textbf{Wysoka dostępność} - redundancja i odporność na awarię dzięki Docker oraz Integracji z VPN.
    \item \textbf{Łatwość w utrzymaniu} - system powinien umozliwiac łatwe aktualizację i rekonfigurację w razie potrzeby ręcznej interwencji.
    \item \textbf{Skalowalność} - mozliwość rozszerzenia o nowe komponenty i usługi.
    \item \textbf{Bezpieczeństwo} - szyfrowanie komunikacji oraz kontrola dostępu do zasobów.
    \item \textbf{Modularność} - podział systemu na niezalezne komponenty działające w kontenerach Docker.
    \item \textbf{Integracja z open-source} - Wspracie dla narzędzi i trchnologii dostępnych na licencji open-source.
    \item \textbf{Minimalizacja kosztów} - niskie koszty sprzętowe i utrzymanie dzieki Raspberry Pi i rozwiązaniom chmurowym typu DuckDNS.
    \item \textbf{Wydajność} - optymalizacja aplikacji pod Raspberry Pi, aby zapenić płynne działąnie 
    \item \textbf{Łatwość wdrozenia} - uproszczona konfiguracja pozwalająca na szybkie uruchomienie systemu.
\end{enumerate}

\pagebreak

\section{Architektura systemu}
System HomeLab składa się z kilku kluczowych komponentów:

\subsection{Backend (FastAPI + MongoDB)}
\begin{itemize}
    \item FastAPI \cite{FastAPI} odpowiada za obsługę logiki biznesowej i API Backendu.
    \item MongoDB \cite{MongoDB} przechowuje dane uzytkowników, konfigurację systemowe i rejestr operacji.
\end{itemize}

\subsection{Frontend (AppSmith)}
\begin{itemize}
    \item niskokodowa platforma pozwalająca na łatwe budowanie interfejsu uzytkownika.
    \item Integracja z backendem poprzez REST API
\end{itemize}

\subsection{Warstwa sieciowa}
\begin{itemize}
    \item Połączenia umozliwione i zabezpieczone poprzez Tailscale (VPN).
    \item DuckDNS zapewniający mozliwość dostępu do systemu za pomocą domeny, zamiast adresu IP.
\end{itemize}

\subsection{Środowisko kontenerowe}
\begin{itemize}
    \item Docker wykorzystywany do zarządzania usługami systemu.
    \item Mozliwość łatwego wdrazania i skalowania aplikacji poprzez kontenery.
\end{itemize}

\subsection{Automatyzacja CI/CD}
\begin{itemize}
    \item GitHub Actions zarządza automatyzacją wdrozeń i aktualizacji systemu
    \item Kazda zmiana w kodzie uruchamia testy oraz wdrozenie nowych wersji aplikacji.
\end{itemize}

\subsection{Urządzenie docelowe}

Raspberry Pi 5 jako główny host systemu, zapewniający niskie zuzycie energii i optymalizację kosztów.

\begin{figure}
    \begin{center}
        \includegraphics[width=0.9\textwidth]{./chapters/mermeid/schemat_architektury.png}
        \caption{Schemat Architektury}
    \end{center}
\end{figure}


\section{Technologie i narzędzia uzyte w systemie}

System HomeLab wykorzystuje następujące technologie:
\subsection{Backend}
\begin{itemize}
    \item FastAPI - szybki i nowoczesny framework do tworzenia API w pythonie
    \item MongoDB - baza danych NoSQL przechowująca konfigurację i dane uzytkowników
\end{itemize}
\subsection{Frontend}
\begin{itemize}
    \item AppSmith - niskokodowe narzędzia do budowy interfejsu uzytkownika.
    \item RestAPI - wykorzystywane do komunikacji między frontendem a backendem.
\end{itemize}
\subsection{Warstwa Sieciowa}
\begin{itemize}
    \item Tailscale - VPN do bezpiecznego zapewnienia zdalnego dostępu do systemu, bez konieczności posiadania stałego adresu IP.
    \item DuckDNS - dynamiczny system zarządzania domeną umozliwiający łatwy dostęp do systemi.
\end{itemize}
\subsection{Środowisko uruchomieniowe}
\begin{itemize}
    \item Docker - uzywany do konteneryzacji aplikacji i zarządzania zaleznościami.
    \item Raspberry Pi 5 - host systemu zapewniający energooszczędność i niski koszt.
\end{itemize}
\subsection{Automatyzacja CI/CD}
\begin{itemize}
    \item GitHub Actions - narzędzie do automatyzacji wdrozeń i testowania kodu.
    \item Pipeline CI/CD - automatyczne testowanie, budowanie i wdrazanie aplikacji
\end{itemize}

Dzięki zastosowaniu powyzszych technologii system Homelab będzie nowoczesnym, skalowalnym i energooszczędnym rozwiązaniem dla uzytkowników domowych.
\chapter{Implementacja systemu}

\section{Backend - API do zarządzania systemem}

\subsection{Struktura API i kluczowe endpointy}

System został zaprojektowany jako aplikacja webowa z backendem opartym na FastAPI\cite{FastAPI}, zapewniającym bezpieczne i efektywne zarządzanie użytkownikami, usługami oraz monitorowanie zasobów systemowych. W niniejszym rozdziale opisano strukturę API oraz kluczowe endpointy wykorzystywane w systemie.

\subsubsection{Architektura API}
Backend aplikacji został zaimplementowany z wykorzystaniem FastAPI \cite{FastAPI}, co zapewnia wysoką wydajność oraz łatwość definiowania i dokumentowania endpointów. API składa się z następujących głównych komponentów:
\begin{itemize}
    \item Autoryzacja użytkowników z wykorzystaniem JWT,
    \item Zarządzanie użytkownikami (dodawanie, edycja, listowanie, usuwanie),
    \item Zarządzanie usługami (rejestracja, uruchamianie, zatrzymywanie, monitorowanie),
    \item Monitorowanie zasobów systemowych (CPU, RAM, kontenery Docker),
    \item Zarządzanie serwerem (aktualizacja, restart, autoryzacja Tailscale, wyłączenie systemu).
\end{itemize}

Każdy z tych komponentów posiada dedykowane endpointy REST API, które są opisane w kolejnych sekcjach.

\subsubsection{Autoryzacja użytkowników}
Autoryzacja odbywa się poprzez mechanizm JWT (JSON Web Token). Kluczowe endpointy:
\begin{itemize}
    \item \textbf{POST /api/register} – Rejestracja nowego użytkownika,
    \item \textbf{POST /api/login} – Logowanie użytkownika i zwrócenie tokena JWT,
    \item \textbf{POST /api/role} – Zmiana roli użytkownika.
\end{itemize}

\subsubsection{Zarządzanie użytkownikami (Admin Only)}
Funkcje zarządzania użytkownikami są dostępne tylko dla administratorów. Kluczowe endpointy:
\begin{itemize}
    \item \textbf{POST /api/users/add} – Dodanie nowego użytkownika,
    \item \textbf{PUT /api/users/edit/\{username\}} – Edycja danych użytkownika,
    \item \textbf{GET /api/users/list} – Pobranie listy wszystkich użytkowników,
    \item \textbf{DELETE /api/users/delete/\{username\}} – Usunięcie użytkownika.
\end{itemize}

\subsubsection{Zarządzanie usługami}
Moduł zarządzania usługami pozwala administratorowi na kontrolowanie działania systemowych procesów. Kluczowe endpointy:
\begin{itemize}
    \item \textbf{POST /api/services/register} – Rejestracja nowej usługi,
    \item \textbf{POST /api/services/start/\{service\_name\}} – Uruchomienie usługi,
    \item \textbf{POST /api/services/stop/\{service\_name\}} – Zatrzymanie usługi,
    \item \textbf{GET /api/services/list} – Pobranie listy aktywnych usług,
    \item \textbf{GET /api/services/health/\{service\_name\}} – Sprawdzenie statusu pojedynczej usługi,
    \item \textbf{GET /api/services/health/all} – Sprawdzenie statusu wszystkich usług,
    \item \textbf{POST /api/services/restart\_unhealthy} – Restartowanie usług w złym stanie.
\end{itemize}

\subsubsection{Monitorowanie systemu}
System umożliwia monitorowanie zasobów serwera oraz uruchomionych kontenerów Docker. Kluczowy endpoint:
\begin{itemize}
    \item \textbf{GET /api/instance/details} – Pobranie szczegółów dotyczących zużycia CPU, RAM oraz listy uruchomionych kontenerów Docker.
\end{itemize}

\subsubsection{Zarządzanie serwerem (Admin Only)}
Administratorzy mogą wykonywać kluczowe operacje na serwerze, takie jak aktualizacje, restart systemu oraz autoryzacja Tailscale. Kluczowe endpointy:
\begin{itemize}
    \item \textbf{POST /api/server/update} – Aktualizacja systemu,
    \item \textbf{POST /api/server/reboot} – Restart serwera,
    \item \textbf{POST /api/server/tailscale-auth} – Autoryzacja Tailscale,
    \item \textbf{POST /api/server/poweroff} – Wyłączenie serwera.
\end{itemize}

\subsubsection{Podsumowanie}
Struktura API została zaprojektowana z myślą o modularności oraz łatwości rozszerzania funkcjonalności. Wykorzystanie FastAPI zapewnia wysoką wydajność oraz wbudowaną dokumentację, co ułatwia integrację z innymi systemami. Mechanizm autoryzacji JWT zapewnia bezpieczeństwo operacji, a administratorzy mogą w prosty sposób zarządzać użytkownikami, usługami oraz operacjami serwera poprzez intuicyjne i dobrze udokumentowane endpointy.


\subsection{Obsługa uwierzytelniania i autoryzacji}

System uwierzytelniania i autoryzacji opiera się na tokenach JWT (JSON Web Token). Mechanizm ten zapewnia bezpieczną kontrolę dostępu do zasobów systemu, umożliwiając przypisanie użytkownikom odpowiednich ról i ograniczenie dostępu do krytycznych operacji administracyjnych.

\subsubsection{Proces uwierzytelniania}
Każdy użytkownik musi zalogować się do systemu, podając swoje dane uwierzytelniające. Po pomyślnej weryfikacji hasła system generuje token JWT, który służy do autoryzacji kolejnych żądań API. Token zawiera informacje o użytkowniku oraz jego roli w systemie.

Kluczowe endpointy:
\begin{itemize}
    \item \textbf{POST /api/register} – Rejestracja nowego użytkownika,
    \item \textbf{POST /api/login} – Logowanie użytkownika i zwrócenie tokena JWT.
\end{itemize}

\section{Frontend - Interfejs uzytkownika}
\includegraphics[width=1\textwidth]{./chapters/assets/user_interface.png}
\subsection{Projekt UI/UX}

Projektowanie interfejsu użytkownika (UI) oraz doświadczenia użytkownika (UX) odgrywa kluczową rolę w zapewnieniu funkcjonalności i intuicyjności systemu. W niniejszym rozdziale omówiono zasady projektowania UI/UX zastosowane w opracowanym interfejsie, bazując na dołączonym projekcie graficznym.

\subsubsection{Założenia projektowe}
Podstawowe cele projektowe interfejsu obejmowały:
\begin{itemize}
    \item Czytelność i prostotę obsługi,
    \item Spójność wizualną oraz intuicyjną nawigację,
    \item Minimalizację liczby kliknięć wymaganych do wykonania operacji,
    \item Odpowiednią organizację informacji w oparciu o hierarchię wizualną.
\end{itemize}

\subsubsection{Struktura interfejsu}
Projekt składa się z kilku głównych obszarów:
\begin{itemize}
    \item \textbf{Panel informacyjny} – zawierający dane dotyczące użycia zasobów systemowych (CPU, RAM, liczba usług).
    \item \textbf{Sekcja zarządzania usługami} – prezentująca status poszczególnych usług (uruchomione, zatrzymane, błędy) w formie wykresu kołowego.
    \item \textbf{Obszar użytkownika} – umożliwiający zarządzanie użytkownikami systemu (dodawanie, usuwanie, edycja).
    \item \textbf{Obszar serwera} – obejmujący podstawowe operacje administracyjne, takie jak aktualizacja systemu, restart oraz autoryzacja w Tailscale.
    \item \textbf{Lista usług} – wyświetlająca poszczególne aplikacje (np. HomeAssistant, Mealie, Paperless, Portainer) z możliwością zarządzania ich statusem poprzez przełączniki.
\end{itemize}

\subsubsection{Kolorystyka i typografia}
Projekt wykorzystuje nowoczesną, stonowaną kolorystykę z przewagą bieli i odcieni fioletu. Przyciski akcji są wyraźnie zaznaczone za pomocą intensywnych kolorów, co zwiększa ich widoczność. Tekst jest wyraźny, a hierarchia informacji zapewnia odpowiednią czytelność.

\subsubsection{Intuicyjność i użyteczność}
Projekt został zaprojektowany zgodnie z zasadami użyteczności:
\begin{itemize}
    \item Przejrzysta organizacja treści,
    \item Spójność w rozmieszczeniu elementów interfejsu,
    \item Minimalizacja zbędnych interakcji,
    \item Responsywność, umożliwiająca korzystanie na różnych urządzeniach.
\end{itemize}

Dzięki tym rozwiązaniom interfejs jest łatwy w obsłudze i spełnia wymagania użytkowników końcowych.


\subsection{Implementacja interfejsu użytkownika z wykorzystaniem podejścia no-code w Appsmith}

W celu usprawnienia procesu tworzenia interfejsu użytkownika wykorzystano platformę no-code Appsmith \cite{AppSmith}, która umożliwia szybkie budowanie aplikacji webowych za pomocą gotowych komponentów. 

\subsubsection{Proces implementacji}
Implementacja interfejsu przebiegała w kilku etapach:
\begin{enumerate}
    \item \textbf{Tworzenie struktury aplikacji} – W Appsmith utworzono nowy projekt, w którym zdefiniowano podstawowe widoki, takie jak pulpit nawigacyjny, sekcja zarządzania użytkownikami oraz obszar monitorowania usług.
    \item \textbf{Dodawanie komponentów UI} – Wykorzystano gotowe komponenty Appsmith, takie jak przyciski, tabele, wykresy i przełączniki do interakcji z użytkownikiem.
    \item \textbf{Konfiguracja źródeł danych} – Interfejs został połączony z backendem poprzez API, co pozwoliło na dynamiczne pobieranie informacji o stanie systemu i usług.
    \item \textbf{Logika aplikacji} – Za pomocą wbudowanego edytora JavaScript skonfigurowano interakcje użytkownika, m.in. obsługę formularzy, wywołania API oraz automatyczne odświeżanie danych.
    \item \textbf{Testowanie i optymalizacja} – Po wdrożeniu interfejsu przeprowadzono testy użyteczności, aby zapewnić płynne działanie aplikacji i zoptymalizować jej wydajność.
\end{enumerate}

\subsubsection{Zalety zastosowania Appsmith}
Wykorzystanie Appsmith przyniosło szereg korzyści w kontekście tworzenia interfejsu użytkownika:
\begin{itemize}
    \item Skrócenie czasu implementacji dzięki gotowym komponentom,
    \item Łatwa integracja z backendem poprzez API,
    \item Możliwość dynamicznej edycji logiki aplikacji bez konieczności pisania pełnego kodu frontendu,
    \item Prosty interfejs edytora umożliwiający szybkie wprowadzanie zmian i testowanie.
\end{itemize}

Zastosowanie podejścia no-code pozwoliło na efektywne wdrożenie interfejsu użytkownika bez konieczności zaawansowanego programowania, co znacząco przyspieszyło proces tworzenia systemu.


\section{Automatyzacja Konfiguracji i wdrozenie}

\subsection{Instalacja rozwiązania}
W ramach tworzenia rozwiązania powstał również program instalacyjny, który zostanie umieszczony na stronie internetowej rozwiązania umożliwiając pobranie przez zainteresowane korzystaniem z rozwiązania osoby.
\paragraph{Opis programu instalacyjnego}

\begin{lstlisting}
package main

import (
	"archive/zip"
	"fmt"
	"io"
	"net/http"
	"os"
	"os/exec"
	"path/filepath"
	"runtime"
)

const (
	releaseURL = "https://github.com/youruser/homelab/releases/latest/download/homelab.zip"
	installDir = "./homelab" // Można zmienić na /opt/homelab lub %ProgramFiles%
)

func main() {
	fmt.Println("🔍 Sprawdzanie Docker...")

	if !isDockerInstalled() {
		fmt.Println("🚧 Docker nie znaleziony. Instaluję...")
		if err := installDocker(); err != nil {
			fmt.Fprintf(os.Stderr, "❌ Błąd instalacji Dockera: %v\n", err)
			return
		}
	} else {
		fmt.Println("✅ Docker jest już zainstalowany.")
	}

	fmt.Println("📦 Pobieranie aplikacji...")
	if err := downloadAndExtract(releaseURL, installDir); err != nil {
		fmt.Fprintf(os.Stderr, "❌ Błąd podczas pobierania/rozpakowywania: %v\n", err)
		return
	}

	fmt.Println("🚀 Uruchamianie aplikacji...")
	if err := runApp(installDir); err != nil {
		fmt.Fprintf(os.Stderr, "❌ Nie udało się uruchomić aplikacji: %v\n", err)
	}
}
\end{lstlisting}

\subsection{Integracja z narzędziami CI/CD}
\label{sec:integracja_ci_cd}

Współczesne systemy informatyczne wymagają nie tylko solidnej implementacji, ale również efektywnego zarządzania cyklem życia oprogramowania. W tym kontekście, integracja narzędzi CI/CD (Continuous Integration / Continuous Deployment) odgrywa kluczową rolę w automatyzacji procesów budowania, testowania i wdrażania aplikacji.

\subsubsection{Zastosowanie self-hosted runnera \cite{SelfHostRunner}}

W celu zapewnienia kompatybilności architektury uruchomieniowej systemu, zdecydowano się na użycie **self-hosted runnera** zamiast domyślnych runnerów GitHub Actions. Domyślne maszyny CI/CD oferowane przez GitHub działają wyłącznie na **architekturze x86**, co ogranicza możliwość testowania i wdrażania aplikacji na innych platformach, takich jak **ARM** (np. Raspberry Pi, serwery oparte na ARM64). 

Zastosowanie własnego runnera pozwala na:
\begin{itemize}
    \item Uruchamianie testów i budowanie obrazów Docker na architekturze zgodnej z docelowym środowiskiem produkcyjnym.
    \item Pełną kontrolę nad zasobami sprzętowymi wykorzystywanymi w procesie CI/CD.
    \item Możliwość integracji z lokalnym registry dla przechowywania obrazów Docker.
\end{itemize}

\subsubsection{Workflow GitHub Actions dla testowania}

\begin{verbatim}
name: Run Tests

on:
  push:
    branches:
      - main
      - dev
  pull_request:

jobs:
  test:
    runs-on: self-hosted

    services:
      mongo:
        image: mongo:latest
        ports:
          - 27017:27017

    steps:
      - name: Checkout Repository
        uses: actions/checkout@v4

      - name: Set Up Python
        uses: actions/setup-python@v4
        with:
          python-version: "3.13"

      - name: Install Dependencies
        run: |
          python -m pip install --upgrade pip
          pip install -r requirements.txt
          pip install pytest pytest-asyncio httpx

      - name: Run Pytest
        run: pytest -v
\end{verbatim}

\subsubsection{Workflow GitHub Actions dla budowania i wersjonowania obrazów Docker}

Po przejściu testów jednostkowych obraz Docker jest budowany i pushowany do lokalnego rejestru uruchomionego na **localhost:5000**.

\begin{verbatim}
name: Build and Push Docker Image

on:
    push:
      branches:
        - main
        - dev
    pull_request:
        branches: [dev, main]

jobs:
  build_and_push:
    runs-on: self-hosted

    steps:
      - name: Checkout Repository
        uses: actions/checkout@v4

      - name: Extract Version
        run: echo "VERSION=$(date +'%Y%m%d')-$(git rev-parse --short HEAD)" >> $GITHUB_ENV

      - name: Build Docker Image
        run: |
          docker build -t localhost:5000/myapp:${{ env.VERSION }} .
          docker tag localhost:5000/myapp:${{ env.VERSION }} localhost:5000/myapp:latest

      - name: Push Docker Image to Local Registry
        run: |
          docker push localhost:5000/myapp:${{ env.VERSION }}
          docker push localhost:5000/myapp:latest
\end{verbatim}

\subsubsection{Korzyści z bezpośredniego pushowania obrazu do registry}

W przeciwieństwie do wcześniejszej konfiguracji, w której obraz był zapisywany lokalnie za pomocą \texttt{docker save}, obecnie jest on od razu przesyłany do prywatnego rejestru. Takie podejście:
\begin{itemize}
    \item Umożliwia natychmiastowe wykorzystanie obrazu w środowisku produkcyjnym bez potrzeby ręcznego jego ładowania.
    \item Pozwala na prostsze zarządzanie wersjami obrazów w registry.
    \item Minimalizuje czas między budowaniem a wdrożeniem.
\end{itemize}

\subsubsection{Publikacja obrazu do lokalnego registry}

Wszystkie wersje aplikacji są automatycznie przechowywane w lokalnym rejestrze **Docker Registry**, a najnowsza wersja jest oznaczana jako \texttt{latest}. Dzięki temu wdrożenie nowej wersji sprowadza się do uruchomienia nowego kontenera:

\begin{verbatim}
docker pull localhost:5000/myapp:latest
docker run -d --name myapp localhost:5000/myapp:latest
\end{verbatim}

Dzięki temu każda wersja aplikacji jest jednoznacznie identyfikowana, a \texttt{latest} wskazuje na najnowszą stabilną wersję.
\chapter{Analiza efektywności i bezpieczeństwa systemu}

\section{Testy jednostkowe i integracyjne}
Testowanie oprogramowania jest kluczowym elementem zapewnienia jego jakości, stabilności i niezawodności. W ramach niniejszej pracy zastosowano zarówno testy jednostkowe, jak i testy integracyjne w celu weryfikacji poprawności działania poszczególnych modułów systemu oraz ich wzajemnych interakcji.

\subsection{Testy jednostkowe}

Testy jednostkowe koncentrują się na sprawdzaniu poprawności działania pojedynczych funkcji i metod w izolacji. Ich głównym celem jest szybkie wykrywanie błędów w logice aplikacji oraz zapewnienie, że każdy komponent działa zgodnie z oczekiwaniami. Do przeprowadzenia testów jednostkowych zastosowano \textbf{go test}

Przykładowe testy jednostkowe obejmowały:
\begin{itemize}
    \item Sprawdzenie poprawności działania funkcji hashującej hasła użytkowników,
    \item Weryfikację generowania i walidacji tokenów JWT,
    \item Testy funkcji odpowiedzialnych za zarządzanie użytkownikami (dodawanie, edycja, usuwanie),
    \item Weryfikację poprawności operacji CRUD dla bazy danych MongoDB.
\end{itemize}

Wszystkie testy jednostkowe zostały zautomatyzowane i uruchamiane w ramach procesu CI/CD z wykorzystaniem GitHub Actions oraz samodzielnie hostowanych runnerów.

Przeprowadzenie testów jednostkowych umożliwiło rozwój serwisu API bez konieczności posiadania gotowego rozwiązania frontend. Dzięki czemu od samego początku możliwa była weryfikacja oraz wychwycenie błędów logicznych w aplikacji. Takie podejście umożliwia szybki rozwój programu bez obawy o działanie pojedynczych metod i funkcji. Testy jednostkowe nie weryfikują całości działania aplikacji a jedynie indywidualnie jej najmniejsze komponenty. Należy mieć na uwadze, że zgodnie ze sztuką testy jednostkowe powinny nie być zależne od innych metod i funkcji a jeśli dana funkcja lub metoda posiada taką zależność należy ją zasymulować w kodzie poprzez stosowanie tzw. \textbf{Mocków}. Jest to nic innego jak symulowanie działania innej funkcji. Umożliwia ono np. w środowisku testowym uzyskiwać odpowiedzi HTTP poprzez zwracanie stałych odpowiedzi bez konieczności wykonywania zapytań HTTP. Pozwala na integrację z plikami bez fizycznej manipulacji plików. Testy jednostkowe to podstawowy element procesu sprawdzania poprawności oprogramowania. Testów tych powinno być najwięcej zgodnie z paradygmem \textbf{Shift Left}.


\subsection{Testy integracyjne}
Testy integracyjne koncentrują się na weryfikacji współpracy między różnymi komponentami systemu. Ich celem jest zapewnienie, że poszczególne moduły działają poprawnie w połączeniu z innymi oraz że komunikacja między nimi przebiega zgodnie z oczekiwaniami. Testy integracyjne są kluczowe dla identyfikacji problemów związanych z interakcjami między komponentami, które mogą nie być widoczne podczas testów jednostkowych.
W ramach testów integracyjnych przeprowadzono:
\begin{itemize}
    \item Weryfikację poprawności działania endpointów API w kontekście komunikacji z bazą danych,
    \item Weryfikację parsowania danych otrzyamnych z zewnętrznych źródeł (np. speedtest) oraz ich poprawnego zapisu do bazy danych,
    \item Testy funkcjonalne dla kluczowych scenariuszy użytkowania, takich jak rejestracja, logowanie, dodawanie usług,
\end{itemize}
Testy integracyjne zostały zautomatyzowane przy użyciu narzędzi takich jak Postman oraz Newman, co umożliwiło ich łatwe uruchamianie i monitorowanie wyników. Dzięki temu możliwe było szybkie wykrywanie problemów związanych z integracją różnych komponentów systemu oraz ich eliminacja.
\section{Analiza wydajnościowa systemu}
\subsection{Testowanie wydajności systemu}

Testowanie wydajności aplikacji jest kluczowe do zweryfikowania zachowania działania programu pod obciążeniem. Przeprowadzenie takiego testu pozwala na znalezienie tzw. "Wąskich gardeł" programu czyli ścieżek logicznych, które najbardziej obciążają działająca aplikację. Dzięki analizie takich testów możemy odpowiednio nadać wagę zadaniom optymalizacyjnym, żeby poprawić odbiór aplikacji poprzez końcowego użytkownika. Testy wydajnościowe to również testy długoterminowe, dzięki nim można zrozumieć jak aplikacja zachowuje się z długotrwałym obciążeniem i poznać jak reaguje na nagłe skoki obciążenia w różnych momentach czasu działania programu.

Do przeprowadzenia testów wydajnościowych użyty został darmowy program \textbf{ddosify} - jego prosta obsługa oraz konfiguracja, pozwoliła na szybkie jego wdrożenie, a przejrzysty raport wykonania testu na sprawną analizę wydajności programu. 

\subsubsection{Cel testów wydajnościowych}
Celem testów wydajnościowych było:
\begin{itemize}
    \item Ocena wydajności API w warunkach wysokiego obciążenia,
    \item Pomiar czasu odpowiedzi kluczowych endpointów,
    \item Weryfikacja stabilności systemu podczas długotrwałego obciążenia,
    \item Identyfikacja potencjalnych wąskich gardeł aplikacji.
    \item Zweryfikowanie czy w aplikacji nie występują wycieki pamięci podczas długotrwałcyh testów.
\end{itemize}

\subsubsection{Metodyka i analiza testów wydajnościowych}

Testy wydajnościowe zostały przeprowadzone w celu oceny stabilności oraz czasu reakcji systemu backendowego pod rosnącym obciążeniem. Do realizacji testów wykorzystano narzędzie \textbf{Ddosify}, które umożliwia generowanie ruchu HTTP z określoną liczbą żądań oraz mierzenie parametrów takich jak średni czas odpowiedzi, liczba błędów oraz przepustowość systemu. Narzędzie to zostało wybrane ze względu na łatwość integracji oraz możliwość uruchamiania testów z poziomu skryptów bashowych.

Celem testów było:
\begin{itemize}
    \item Określenie, jak system reaguje na wzrastające obciążenie,
    \item Pomiar średniego czasu odpowiedzi dla różnych poziomów liczby zapytań,
    \item Weryfikacja stabilności systemu i pojawiających się błędów przy dużym ruchu,
    \item Wskazanie potencjalnych miejsc wymagających optymalizacji.
\end{itemize}

W ramach testu wykonano serię zapytań do jednego z endpointów API, zwiększając liczbę żądań w krokach od 50 do 500. Dla każdego poziomu obciążenia zarejestrowano średni czas odpowiedzi. Dane zostały zapisane i zwizualizowane w postaci wykresu.

\begin{figure}[H]
    \centering
    \includegraphics[width=0.9\textwidth]{chapters/assets/wykres czasu odpowiedzi.png}
    \caption{Średni czas odpowiedzi API w zależności od liczby żądań}
\end{figure}

Na powyższym wykresie zauważalna jest względna stabilność średniego czasu odpowiedzi dla zakresu od 50 do około 300 zapytań. W tym przedziale wartości oscylowały wokół 0.18-0.22 ms. Po przekroczeniu 300 żądań widoczny jest wyraźny wzrost opóźnień, a największy skok nastąpił przy 450 żądaniach, gdzie średni czas odpowiedzi osiągnął ponad 0.75 ms. W ostatniej próbie (500 żądań) wszystkie zapytania zakończyły się błędem (500/500 błędów), co skutkowało średnim czasem odpowiedzi równym 0 ms, gdyż odpowiedzi nie zostały zwrócone w ogóle lub zakończyły się błędami.

Za główną przyczynę wystąpienia błędów uznano ograniczenia wydajnościowe bazy danych SQLite, która nie jest przystosowana do obsługi wielu równoczesnych zapytań w środowiskach wysokiego obciążenia. W szczególności brak wsparcia dla wielu jednoczesnych zapisów prowadził do blokowania operacji i przekroczenia limitów czasu odpowiedzi. Dodatkowym czynnikiem była ograniczona liczba zasobów przydzielona aplikacji w środowisku testowym.

\textbf{Wnioski:}
\begin{itemize}
    \item Aplikacja działa stabilnie przy niskim i umiarkowanym obciążeniu (do 300 równoczesnych zapytań),
    \item Wzrost liczby równoczesnych żądań znacząco wpływa na opóźnienia odpowiedzi, co wskazuje na konieczność optymalizacji backendu,
    \item Wystąpienie 100\% błędów przy maksymalnym obciążeniu (500 zapytań) wskazuje na poważne ograniczenie w warstwie komunikacji z bazą danych.
\end{itemize}

\textbf{Rekomendowane działania optymalizacyjne:}
\begin{itemize}
    \item Zastosowanie mechanizmu \textit{connection poolingu} oraz zwiększenie maksymalnej liczby jednoczesnych połączeń do bazy danych,
    \item Wprowadzenie cache'owania odpowiedzi API dla często wywoływanych zapytań,
    \item Refaktoryzacja zapytań do bazy danych - redukcja złożoności oraz optymalizacja indeksów,
    \item Przeprowadzenie testów w środowisku zbliżonym do produkcyjnego z większą wydajnością I/O bazy danych.
\end{itemize}

\noindent
Należy zaznaczyć, że testy wydajnościowe w przedstawionej formie przypominają bardziej atak typu DDoS niż rzeczywiste wykorzystanie systemu przez użytkowników. Mimo to są one niezwykle przydatne w procesie optymalizacji aplikacji - pozwalają zidentyfikować komponenty wymagające poprawy wydajnościowej. Nawet jeśli dana optymalizacja może wydawać się zbędna na etapie prototypowania, warto mieć na uwadze, że zoptymalizowany kod zużywa mniej zasobów systemowych i lepiej skaluje się w środowiskach produkcyjnych.

\section{Analiza odporności systemu na zagrożenia bezpieczeństwa}

Bezpieczeństwo aplikacji webowych jest jednym z kluczowych aspektów projektowania i wdrażania nowoczesnych systemów. W kontekście niniejszego projektu, który zakłada zarządzanie usługami i użytkownikami, istotne jest zapewnienie odpowiedniego poziomu zabezpieczeń, aby zapobiec nieautoryzowanemu dostępowi, manipulacji danymi oraz wyciekom informacji.

\subsection{Zakres testów bezpieczeństwa}

Testowanie bezpieczeństwa przeprowadzono w ograniczonym zakresie, skupiając się na ręcznej weryfikacji najczęściej występujących zagrożeń w aplikacjach webowych, zgodnie z zestawieniem OWASP Top 10. Weryfikacji poddano następujące obszary:

\paragraph{1. Uwierzytelnianie i autoryzacja}
System wykorzystuje tokeny JWT jako mechanizm autoryzacji. W ramach testów:
\begin{itemize}
    \item Sprawdzono odporność logowania na ataki brute-force - nie wykryto możliwości obejścia zabezpieczeń, jednak brak limitu prób logowania może stanowić potencjalne ryzyko.
    \item Zweryfikowano, że po wygaśnięciu tokena nie jest on akceptowany przez serwer.
    \item Potwierdzono, że użytkownicy nieautoryzowani nie mają dostępu do zasobów innych użytkowników.
\end{itemize}

\paragraph{2. Odporność na SQL Injection}
Wszystkie zapytania do bazy danych wykonywane są za pomocą ORM (GORM), co zapewnia parametryzację i zmniejsza ryzyko podatności na SQL Injection. Próby wstrzyknięcia kodu SQL w pola formularzy zakończyły się niepowodzeniem - aplikacja poprawnie odrzucała nieprawidłowe dane wejściowe.

\paragraph{3. Odporność na XSS i CSRF}
\begin{itemize}
    \item Testy XSS (reflected i stored) nie przyniosły rezultatów - aplikacja nie wstrzykuje niezweryfikowanych danych wejściowych do HTML.
    \item Aplikacja wykorzystuje token JWT w nagłówkach, co redukuje ryzyko CSRF, ponieważ brak ciasteczek sesyjnych eliminuje możliwość automatycznego przesyłania uwierzytelnienia.
\end{itemize}

\paragraph{4. Bezpieczeństwo API}
Przeprowadzono testy zabezpieczeń API:
\begin{itemize}
    \item Wszystkie endpointy wymagające autoryzacji poprawnie zwracały kod 401 w przypadku braku lub niewłaściwego tokena.
    \item System nie pozwalał na wykonanie operacji z tokenem innego użytkownika.
    \item Weryfikowano także poprawność nagłówków CORS oraz odpowiedzi serwera na próby nieautoryzowanego dostępu.
\end{itemize}

\paragraph{5. Odporność na przeciążenie (DoS)}
W ramach testów wydajnościowych przeprowadzono symulację ataku DoS:
\begin{itemize}
    \item Wysyłano dużą liczbę zapytań w krótkim czasie - powyżej 400 jednoczesnych zapytań powodowało błędy odpowiedzi serwera.
    \item Błędy te były spowodowane ograniczeniami bazy SQLite, która nie wspiera wielu równoczesnych zapisów.
\end{itemize}

\subsection{Rekomendacje i dalsze działania}
Choć aplikacja wykazała odporność na najczęstsze zagrożenia, zaleca się:
\begin{itemize}
    \item Dodanie limitu prób logowania i mechanizmu blokowania IP,
    \item Wdrożenie pełnych testów penetracyjnych (np. z użyciem Burp Suite \cite{BurpSuite} lub OWASP ZAP\cite{ZAP}),
    \item Monitorowanie logów bezpieczeństwa i integracja z systemem alertów,
    \item Przejście na bardziej wydajną bazę danych, wspierającą konkurencyjne zapisy (np. PostgreSQL) w przypadku planowanej produkcyjnej eksploatacji systemu.
\end{itemize}

\textbf{Podsumowanie:} Przeprowadzone testy bezpieczeństwa potwierdzają, że aplikacja w obecnym stanie spełnia podstawowe wymogi bezpieczeństwa, w tym poprawną autoryzację, separację danych użytkowników oraz odporność na podstawowe ataki webowe, takie jak XSS, CSRF czy SQL Injection. Skuteczna implementacja mechanizmu JWT, ochrona endpointów oraz brak dostępu do zasobów bez ważnego tokena świadczą o przemyślanej strukturze kontroli dostępu. Niemniej jednak, dalszy rozwój aplikacji - zwłaszcza w kontekście wdrożenia produkcyjnego - powinien uwzględniać wdrażanie zaawansowanych technik testowania i obrony, takich jak automatyczne testy penetracyjne, analiza statyczna kodu źródłowego, monitorowanie logów pod kątem prób ataków oraz integracja z systemami SIEM (Security Information and Event Management). Rekomendowane jest również regularne wykonywanie audytów bezpieczeństwa oraz testów regresyjnych po każdej większej zmianie kodu lub zależności. Tylko takie podejście pozwoli zapewnić długofalowe bezpieczeństwo systemu oraz ochronę danych użytkowników.
\chapter{Podsumowanie i wnioski}

\section{Podsumowanie pracy}

W niniejszej pracy magisterskiej przedstawiono projekt, implementację oraz analizę systemu zarządzania usługami i monitorowania serwera, napisanego w języku Go. System składa się z backendu (API) stworzonego w Go oraz frontendowej części zbudowanej w oparciu o szablon TailAdmin. Celem pracy było stworzenie wydajnej i bezpiecznej aplikacji umożliwiającej administratorom zarządzanie usługami systemowymi, monitorowanie wykorzystania zasobów oraz przeprowadzanie operacji administracyjnych na serwerze.

W trakcie realizacji projektu skupiono się na kilku kluczowych aspektach, w tym implementacji API w Go, integracji z interfejsem użytkownika TailAdmin, zarządzaniu użytkownikami, optymalizacji wydajności oraz aspektach związanych z bezpieczeństwem aplikacji webowych. Każdy z tych elementów został szczegółowo opisany i przeanalizowany, co pozwoliło na wyciągnięcie cennych wniosków dotyczących zarówno języka Go, jak i nowoczesnych narzędzi frontendowych i backendowych stosowanych w zarządzaniu infrastrukturą IT.

Instrukcja instalacji oraz uruchomienia aplikacji została zamieszczona w pliku \texttt{README.md} w repozytorium projektu na platformie GitHub\cite{NestOpsV2}.


\subsection{Osiągnięcia i rezultaty pracy}

W ramach realizacji pracy magisterskiej osiągnięto szereg istotnych rezultatów, które potwierdzają praktyczną użyteczność oraz skalowalność zaprojektowanego systemu. Najważniejsze z nich to:

\begin{itemize}
    \item Zaprojektowano oraz zaimplementowano backend systemu przy użyciu języka Go. Dzięki zastosowaniu tej technologii aplikacja cechuje się wysoką wydajnością, prostotą dystrybucji oraz łatwością utrzymania.
    \item Opracowano nowoczesny, responsywny interfejs użytkownika wykorzystując szablon TailAdmin. Interfejs ten umożliwia intuicyjne zarządzanie usługami systemowymi i użytkownikami, a także prezentuje dane o stanie systemu w przejrzystej formie.
    \item Stworzono kompletne REST API umożliwiające zarządzanie użytkownikami, usługami oraz monitorowanie parametrów systemowych. API zostało zaprojektowane w sposób modularny i rozszerzalny.
    \item Wdrożono mechanizmy uwierzytelniania i autoryzacji użytkowników z wykorzystaniem JWT. System obsługuje różne poziomy uprawnień oraz zabezpiecza dostęp do krytycznych funkcjonalności.
    \item Zaimplementowano funkcjonalności administracyjne obejmujące dodawanie, edytowanie, usuwanie i przeglądanie użytkowników z poziomu panelu administratora.
    \item Umożliwiono rejestrowanie i zarządzanie usługami systemowymi – uruchamianie, zatrzymywanie oraz monitorowanie ich stanu w czasie rzeczywistym.
    \item Zintegrowano backend z systemem operacyjnym w zakresie kontroli nad aktywnymi usługami systemowymi i dostępem do danych o zużyciu zasobów (RAM, CPU, dysk).
    \item Przeprowadzono testy wydajnościowe z użyciem narzędzia Ddosify, które wykazały odporność aplikacji na duże obciążenia oraz pomogły zidentyfikować krytyczne momenty wymagające optymalizacji.
    \item Zrealizowano testy bezpieczeństwa, w tym odporności na ataki typu brute-force, dostęp nieautoryzowany, SQL Injection oraz Cross-Site Scripting, potwierdzając bezpieczeństwo podstawowych mechanizmów aplikacji.
    \item Opracowano system budowania i pakowania aplikacji za pomocą Makefile, umożliwiający łatwą kompilację na wiele platform oraz automatyczne tworzenie paczek instalacyjnych.
    \item Udokumentowano projekt w postaci pliku \texttt{README.md} na GitHubie, w którym zawarto instrukcję instalacji, konfiguracji i uruchomienia systemu zarówno na lokalnych maszynach, jak i w środowiskach produkcyjnych.
\end{itemize}

System powstały w wyniku realizacji pracy magisterskiej został zaprojektowany zgodnie z zasadami czystej architektury i modularności, co pozwala na jego dalsze rozwijanie i dostosowywanie do indywidualnych potrzeb użytkowników oraz specyfiki środowiska produkcyjnego. Aktualna wersja spełnia wszystkie założenia funkcjonalne, stanowiąc stabilną podstawę dla rzeczywistego wdrożenia.

\subsection{Wnioski i przyszłe kierunki rozwoju}

Na podstawie przeprowadzonych badań, testów oraz implementacji można sformułować następujące wnioski:

\begin{itemize}
    \item Połączenie języka Go w warstwie backendu oraz szablonu TailAdmin w warstwie frontendowej pozwala na szybkie tworzenie aplikacji o wysokiej wydajności i przejrzystym interfejsie użytkownika.
    \item Architektura systemu została zaprojektowana w sposób umożliwiający jego łatwą rozbudowę – zarówno poprzez dodawanie nowych endpointów API, jak i integrację z zewnętrznymi usługami.
    \item Testy wydajnościowe ujawniły, że system dobrze radzi sobie z typowym obciążeniem, jednak należy zwrócić uwagę na ograniczenia wynikające z zastosowania SQLite, szczególnie przy dużej liczbie jednoczesnych operacji zapisu.
    \item Mechanizmy bezpieczeństwa wdrożone w systemie (autoryzacja JWT, kontrola ról, odporność na podstawowe ataki webowe) są wystarczające na etapie prototypowania, jednak wymagają dalszej rozbudowy przed wdrożeniem produkcyjnym.
\end{itemize}

W przyszłości system może zostać rozszerzony o:

\begin{itemize}
    \item Wdrożenie kolejki zadań (np. z użyciem Redis lub RabbitMQ) w celu obsługi operacji asynchronicznych oraz przetwarzania zadań wymagających więcej czasu.
    \item Rozszerzenie API o nowe funkcjonalności, takie jak integracja z narzędziami CI/CD, mechanizmy webhooków, czy eksport danych do zewnętrznych systemów.
    \item Zastosowanie bazy danych wspierającej jednoczesne zapisy (np. PostgreSQL) dla zwiększenia skalowalności.
    \item Integrację z narzędziami monitorującymi (np. Prometheus, Grafana) i systemami SIEM w celu zwiększenia widoczności operacyjnej i poziomu bezpieczeństwa.
    \item Wprowadzenie funkcjonalności cache’owania danych, aby zmniejszyć obciążenie backendu i przyspieszyć odpowiedzi aplikacji.
    \item Wdrożenie mechanizmów uwierzytelniania wieloskładnikowego (MFA) oraz integrację z zewnętrznymi systemami tożsamości (np. LDAP, OAuth).
\end{itemize}

Podsumowując, opracowana aplikacja stanowi solidne i nowoczesne rozwiązanie, które może być wykorzystane zarówno w małych środowiskach serwerowych, jak i jako baza pod rozbudowane systemy administracji IT. Projekt może zostać rozwinięty o nowe funkcje i zintegrowany z narzędziami stosowanymi w profesjonalnym zarządzaniu infrastrukturą informatyczną.


\section{Możliwości dalszego rozwoju systemu}

Opracowany system zarządzania usługami i monitorowania serwera został zaprojektowany w sposób modularny i skalowalny, co pozwala na jego dalszą rozbudowę. Możliwe kierunki rozwoju obejmują zarówno ulepszenia funkcjonalne, jak i wdrożenie zaawansowanych mechanizmów zwiększających wydajność oraz bezpieczeństwo systemu. W niniejszym rozdziale przedstawiono propozycje rozszerzeń, które mogą znacząco zwiększyć wartość aplikacji w praktycznym zastosowaniu.

\subsection{Rozszerzenie funkcjonalności zarządzania usługami}
Jednym z kluczowych aspektów rozwoju systemu jest zwiększenie możliwości zarządzania usługami. W obecnej wersji administratorzy mogą rejestrować, uruchamiać, zatrzymywać i monitorować usługi. Można jednak wprowadzić dodatkowe funkcjonalności, takie jak:
\begin{itemize}
    \item \textbf{Automatyczna rekonfiguracja usług} – możliwość dynamicznego dostosowywania parametrów działania usług na podstawie monitorowanych wskaźników wydajności,
    \item \textbf{Harmonogramowanie zadań} – funkcja umożliwiająca administratorom zaplanowanie uruchamiania lub restartowania usług w określonych przedziałach czasowych,
    \item \textbf{Rejestrowanie logów systemowych} – pełna historia zmian w stanie usług wraz z integracją z narzędziami do analizy logów (np. ELK Stack),
    \item \textbf{Automatyczna naprawa błędów} – system wykrywania i samonaprawy usług w przypadku wykrycia awarii.
\end{itemize}

\subsection{Zaawansowane mechanizmy monitorowania}
Obecnie system pozwala na podstawowe monitorowanie wykorzystania CPU, pamięci RAM oraz aktywnych usług. Możliwości dalszego rozwoju obejmują:
\begin{itemize}
    \item \textbf{Monitorowanie wykorzystania sieci} – analiza ruchu sieciowego generowanego przez poszczególne usługi,
    \item \textbf{Alerty i powiadomienia} – implementacja powiadomień o przekroczeniu krytycznych wartości obciążenia systemu (NTFY \cite{NTFY}, Slack \cite{Slack}, Telegram \cite{Telegram}),
    \item \textbf{Predykcja awarii} – zastosowanie algorytmów uczenia maszynowego do przewidywania potencjalnych awarii na podstawie analizy historycznych danych,
    \item \textbf{Dashboard w czasie rzeczywistym} – interaktywna wizualizacja danych systemowych z aktualizacją w czasie rzeczywistym,
    \item \textbf{Integracja z Prometheus i Grafana} – zaawansowane narzędzia monitorowania umożliwiające gromadzenie metryk i ich wizualizację.
\end{itemize}

\subsection{Optymalizacja wydajności systemu}
Testy wydajnościowe wykazały, że system działa sprawnie, ale jego dalszy rozwój może skupić się na:
\begin{itemize}
    \item \textbf{Implementacji cache'owania danych} – redukcja liczby zapytań do bazy danych i API poprzez zastosowanie Redis lub dekoratora cache,
    \item \textbf{Asynchronicznego przetwarzania operacji} – wdrożenie systemu kolejkowania zadań (np. Celery) dla długotrwałych operacji,
    \item \textbf{Load Balancing} – równoważenie obciążenia poprzez podział ruchu między wiele instancji serwera API,
    \item \textbf{Obsługa kontenerów} – rozszerzenie systemu o pełne zarządzanie kontenerami Docker, w tym automatyczne skalowanie usług w zależności od obciążenia.
\end{itemize}

\subsection{Zaawansowane mechanizmy bezpieczeństwa}
Chociaż w pracy omówiono teoretyczne aspekty bezpieczeństwa, możliwe są dodatkowe usprawnienia:
\begin{itemize}
    \item \textbf{Wdrożenie uwierzytelniania wieloskładnikowego (MFA)} – zwiększenie poziomu bezpieczeństwa użytkowników,
    \item \textbf{Rozszerzone uprawnienia użytkowników} – możliwość nadawania niestandardowych ról z precyzyjnie określonymi uprawnieniami,
    \item \textbf{Audyt logów i analiza zachowań} – monitorowanie aktywności użytkowników oraz automatyczne wykrywanie podejrzanych działań,
    \item \textbf{Automatyczne skanowanie podatności} – integracja z narzędziami do analizy bezpieczeństwa, np. OWASP ZAP czy Nessus,
    \item \textbf{Szyfrowanie danych wrażliwych} – wdrożenie szyfrowania kluczowych danych przechowywanych w systemie.
\end{itemize}

\subsection{Integracja z innymi systemami}
W celu zwiększenia użyteczności systemu warto rozważyć jego integrację z innymi rozwiązaniami IT, np.:
\begin{itemize}
    \item \textbf{Integracja z Active Directory / LDAP} – centralne zarządzanie użytkownikami i uprawnieniami,
    \item \textbf{Integracja z systemami DevOps} – połączenie z narzędziami CI/CD (Jenkins, GitHub Actions) w celu automatyzacji wdrożeń,
    \item \textbf{API publiczne} – umożliwienie innym systemom korzystania z funkcjonalności poprzez bezpieczne, udokumentowane API,
    \item \textbf{Obsługa wielu serwerów} – rozbudowa systemu do zarządzania wieloma maszynami w ramach jednej platformy,
    \item \textbf{Integracja z chmurą} – możliwość wdrożenia systemu w chmurach publicznych (AWS, Azure, GCP) i zarządzania zasobami.
\end{itemize}

\subsection{Podsumowanie}
Proponowane kierunki rozwoju pokazują szerokie możliwości dalszej rozbudowy systemu. Obecna wersja stanowi solidną bazę do wprowadzania nowych funkcjonalności, zarówno pod kątem optymalizacji działania, jak i zwiększenia bezpieczeństwa oraz zakresu zastosowań. W przyszłości system może ewoluować w kierunku kompleksowego narzędzia do zarządzania infrastrukturą IT, łącząc aspekty monitorowania, automatyzacji i bezpieczeństwa w jednym rozwiązaniu. Dzięki elastycznej architekturze opartej na API wykonanym w GO GIN oraz szablonu frontendu TailAdmin, aplikacja jest gotowa na integrację z innymi systemami i dalszy rozwój w zależności od potrzeb użytkowników.

% Bibliography
\newpage
\printbibliography

\end{document}